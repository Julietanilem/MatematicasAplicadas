\documentclass[12pt]{article}

\usepackage[spanish]{babel}

\usepackage{amsmath}
\usepackage{amssymb}

\usepackage{hyperref}
\usepackage{graphicx}
\usepackage{listings}
\usepackage{color}
\usepackage{multicol}
\usepackage{enumitem}
\usepackage{here}
\usepackage{dsfont}
\usepackage{tipa}
\usepackage{float}
\usepackage{dsfont} 
\spanishdecimal{.}

\title{Matemáticas para las Ciencias Aplicadas II}
\title{
        \textbf{Tarea 04} \\
        \vspace{1ex}
        \large Matemáticas para las Ciencias Aplicadas II \\
        Facultad de Ciencias, UNAM}
\date{\today}
\author{Flores Morán Julieta Melina \\ Zarco Romero José Antonio}

\begin{document}
\maketitle

% 1 -------------------------------------------------------------------------------------------------------------
\section{}
Verifique que la función $z = \ln{[e^x + e^y]}$ es una solución de las ecuaciones diferenciales:

\begin{itemize}[format=\textbf]

\item $\frac{\partial z}{\partial x} + \frac{\partial z}{\partial y} = 1$

  Primero calculamos las derivadas parciales de primer orden necesarias:

  Para $\frac{\partial z}{\partial x}$
  \begin{align*}
    \frac{\partial z}{\partial x}
    &= \frac{1}{e^x+e^y} \cdot \frac{\partial}{\partial x} (e^x+e^y) \\
    &= \frac{1}{e^x+e^y} \cdot e^x \\
    &= \frac{e^x}{e^x+e^y}
  \end{align*}

  Para $\frac{\partial z}{\partial y}$
  \begin{align*}
    \frac{\partial z}{\partial x}
    &= \frac{1}{e^x+e^y} \cdot \frac{\partial}{\partial y} (e^x+e^y) \\
    &= \frac{1}{e^x+e^y} \cdot e^y \\
    &= \frac{e^y}{e^x+e^y}
  \end{align*}

  Así, $\frac{\partial z}{\partial x} + \frac{\partial z}{\partial y}$
  \begin{align*}
    \frac{\partial z}{\partial x} + \frac{\partial z}{\partial y} 
    &= \frac{e^x}{e^x+e^y} + \frac{e^y}{e^x+e^y} \\
    &= \frac{e^x + e^y}{e^x+e^y} \\
    &= 1
  \end{align*}

  $\therefore z = \ln{[e^x + e^y]}$ satisface la ecuación $\frac{\partial z}{\partial x} + \frac{\partial z}{\partial y} = 1$.

\item $\frac{\partial^2 z}{\partial x^2} \frac{\partial^2 z}{\partial y^2} - \left( \frac{\partial^2 z}{\partial x \partial y} \right)^2=0$

  Primero, a partir de $\frac{\partial z}{\partial x}=\frac{e^x}{e^x+e^y}$ y $\frac{\partial z}{\partial y}=\frac{e^y}{e^x+e^y}$, calculamos las derivadas parciales de segundo orden necesarias:

  Para $\frac{\partial^2 z}{\partial^2 x}$
  \begin{align*}
    \frac{\partial^2 z}{\partial^2 x}
    &= \frac{\partial}{\partial x} \left( \frac{e^x}{e^x+e^y} \right)\\
    &= \frac{[(e^x+e^y)\cdot e^x]-[e^x \cdot e^x]}{(e^x+e^y)^2} \\
    &= \frac{e^{2x}+e^{x+y}-e^{2x}}{(e^x+e^y)^2}\\
    &= \frac{e^{x+y}}{(e^x+e^y)^2}\\
  \end{align*}

  Para $\frac{\partial^2 z}{\partial^2 y}$
  \begin{align*}
    \frac{\partial^2 z}{\partial^2 y}
    &= \frac{\partial}{\partial y} \left( \frac{e^y}{e^x+e^y} \right)\\
    &= \frac{[(e^x+e^y)\cdot e^y]-[e^y \cdot e^y]}{(e^x+e^y)^2} \\
    &= \frac{e^{x+y}+e^{2y}-e^{2y}}{(e^x+e^y)^2}\\
    &= \frac{e^{x+y}}{(e^x+e^y)^2}\\
  \end{align*}

  Para $\frac{\partial^2 z}{\partial x \partial y}$
  \begin{align*}
    \frac{\partial^2 z}{\partial x \partial y}
    &= \frac{\partial}{\partial x} \left( \frac{\partial z}{\partial y} \right) \\
    &= \frac{\partial}{\partial x} \left( \frac{e^y}{e^x+e^y} \right) \\
    &= \frac{[(e^x+e^y)\cdot 0] - (e^y \cdot e^x)}{(e^x+e^y)^2} \\
    &= - \frac{e^{x+y}}{(e^x+e^y)^2}\\
  \end{align*}

  Así, $\frac{\partial^2 z}{\partial x^2} \frac{\partial^2 z}{\partial y^2} - \left( \frac{\partial^2 z}{\partial x \partial y} \right)^2$
  \begin{align*}
    \frac{\partial^2 z}{\partial x^2} \frac{\partial^2 z}{\partial y^2} - \left( \frac{\partial^2 z}{\partial x \partial y} \right)^2
    &= \left[ \frac{e^{x+y}}{(e^x+e^y)^2}\cdot \frac{e^{x+y}}{(e^x+e^y)^2} \right] - \left[- \frac{e^{x+y}}{(e^x+e^y)^2}\right]^2 \\
    &= \frac{(e^{x+y})^2}{(e^{x+y})^4} - \frac{(e^{x+y})^2}{(e^{x+y})^4} \\
    &= 0
  \end{align*}

  $\therefore z = \ln{[e^x + e^y]}$ satisface la ecuación $\frac{\partial^2 z}{\partial x^2} \frac{\partial^2 z}{\partial y^2} - \left( \frac{\partial^2 z}{\partial x \partial y} \right)^2=0$.

\end{itemize}

% 2 -------------------------------------------------------------------------------------------------------------
\section{}

La \textit{energía cinética} de un cuerpo de masa $m$ y velocidad $v$ es $K = \frac{1}{2}mv^2$. Demuestre que $K=\frac{\partial K}{\partial m} \frac{\partial ^2 K}{\partial v^2}$. \\

Primero calculamos las derivadas parciales de primer orden necesarias:

Para $\frac{\partial K}{\partial m}$
\begin{align*}
  \frac{\partial K}{\partial m}
  &= \frac{1}{2} \left(\frac{\partial}{\partial m}mv^2 \right) \\
  &= \frac{1}{2} v^2
\end{align*}

Para $\frac{\partial K}{\partial v}$
\begin{align*}
  \frac{\partial K}{\partial v}
  &= \frac{1}{2} \left(\frac{\partial}{\partial v}mv^2 \right) \\
  &= \frac{1}{2} \cdot 2vm \\
  &= vm
\end{align*}

Ahora, calculamos la derivada parcial de segundo orden necesaria:

Para $\frac{\partial K}{\partial v^2}$
\begin{align*}
  \frac{\partial K}{\partial v^2}
  &= \frac{\partial K}{\partial v} \left( vm \right) \\
  &= m
\end{align*}

Así, $\frac{\partial K}{\partial m} \frac{\partial ^2 K}{\partial v^2}$
\begin{align*}
  \frac{\partial K}{\partial m} \frac{\partial ^2 K}{\partial v^2}
  &= \frac{1}{2} v^2 \cdot m \\
  &= \frac{1}{2} mv^2 \\
  &= K
\end{align*}

$\therefore K=\frac{1}{2} mv^2$ satisface la ecuación $K=\frac{\partial K}{\partial m} \frac{\partial ^2 K}{\partial v^2}$.

\begin{flushright}
  $\blacksquare$
\end{flushright}

% 3 -------------------------------------------------------------------------------------------------------------
\section{}

Determine una ecuación del plano tangente a la función $z = xe^{xy}$ en el punto $(x_0, y_0) = (5, 0)$. \\

Sea $z = xe^{xy}$. Entonces

\begin{align*}
  f_x(x,y)
  &= x\frac{\partial}{\partial x}e^{xy} + e^{xy}\frac{\partial}{\partial x}x \\
  &= xye^{xy} + e^{xy} \\
  &= (xy+1)e^{xy} \\
  f_x(5,0)
  &= (5\cdot 0 + 1) e^{5\cdot 0} \\
  &= 1 \cdot e^0 \\
  &= 1
\end{align*}

\begin{align*}
  f_y(x,y)
  &= x\frac{\partial}{\partial y}e^{xy} + e^{xy}\frac{\partial}{\partial y}x \\
  &= x^2e^{xy}+0 \\
  &= x^2e^{xy} \\
  f_y(5,0)
  &= 5^2\cdot e^{5\cdot 0} \\
  &= 5^2\cdot e^0 \\
  &= 25 \cdot 1 \\
  &= 25
\end{align*}

Dado que $x=5$ y $y=0$, se tiene que $z=5\cdot e^{5\cdot 0} = 5\cdot e^0=5$
Entonces, da la ecuación del plano tangente en $(5,0,5)$ como
$$z-5=1(x-5)+25(y-0)$$
o bien,
$$z=x+25y$$

% 4 -------------------------------------------------------------------------------------------------------------
\section{}

Compruebe que la \textbf{aproximación lineal} en (0, 0).
$$\frac{2x+3}{4y +1} \approx 2x - 12y + 3$$ \\

Sea $f(x,y)=\frac{2x+3}{4y +1}$. Tenemos que las derivadas parciales son

\begin{align*}
  f_x(x,y)
  &= \frac{\left[(4y +1)\cdot \frac{\partial}{\partial x} (2x+3)\right] - \left[(2x+3)\frac{\partial}{\partial x}(4y+1)\right]}{(4y +1)^2} \\
  &= \frac{2(4y+1)-0}{(4y +1)^2} \\
  &= \frac{2(4y+1)}{(4y +1)(4y+1)} \\
  &= \frac{2}{4y +1} \\
  f_x(0,0) &= 2 \\ \\
  f_y(x,y)
  &= \frac{\left[(4y +1)\cdot \frac{\partial}{\partial y} (2x+3)\right] - \left[(2x+3)\frac{\partial}{\partial y}(4y+1)\right]}{(4y +1)^2} \\
  &= \frac{0-4(2x+3)}{(4y +1)^2} \\
  &= \frac{-8x-12}{(4y +1)^2} \\
  f_y(0,0)&=-12
\end{align*}

Tanto $f_x$ como $f_y$ son continuas y existen cerca de $(0,0)$, de modo que $f$ es diferenciable en $(0,0)$. La linealización es

\begin{align*}
  L(x,y)
  &= f(0,0) + 2(x-0) + (-12)(y-0) \\
  &= 2x -12y + 3
\end{align*}

% 5 -------------------------------------------------------------------------------------------------------------
\section{}

Utilice la \textbf{regla de la cadena} para calcular $\frac{\partial z}{\partial s}$ y$\frac{\partial z}{\partial t}$. Dado que
$$z = \sin{\theta}\cos{\phi}, ~ \theta = st^2, ~ \phi = s^2t$$ 

Primero calculamos las derivadas parciales de primer orden necesarias:

\begin{align*}
  \frac{\partial z}{\partial \theta}
  &= \frac{\partial}{\partial \theta} (\sin{\theta}\cos{\phi}) \\
  &= \left(\sin{\theta}\cdot\frac{\partial}{\partial \theta}\cos{\phi}\right) + \left(\cos{\phi}\cdot\frac{\partial}{\partial \theta}\sin{\theta}\right)\\
  &= 0 + \cos{\phi}\cos{\theta} \\
  &= \cos{\theta}\cos{\phi} \\
  \frac{\partial z}{\partial \phi}
  &= \frac{\partial}{\partial \phi}( \sin{\theta}\cos{\phi}) \\
  &= \left(\sin{\theta}\cdot\frac{\partial}{\partial \phi}\cos{\phi}\right) + \left(\cos{\phi}\cdot\frac{\partial}{\partial \phi}\sin{\theta}\right)\\
  &= (\sin{\theta} \cdot -\sin{\phi}) + 0 \\
  &= -\sin{\theta}\sin{\phi}
\end{align*}

\begin{align*}
  \frac{\partial \theta}{\partial s}
  &= \frac{\partial}{\partial s}( st^2) \\
  &= \left(s\cdot\frac{\partial}{\partial s} t^2\right) + \left(t^2\cdot\frac{\partial}{\partial s}s\right) \\
  &= 0 + t^2 \\
  &= t^2 \\
  \frac{\partial \phi}{\partial s}
  &= \frac{\partial}{\partial s} (s^2t) \\
  &= \left(s^2\cdot\frac{\partial}{\partial s}t\right)+\left(t\cdot\frac{\partial}{\partial s}s^2\right) \\
  &= 0 + 2st \\
  &= 2st
\end{align*}

\begin{align*}
  \frac{\partial \theta}{\partial t}
  &= \frac{\partial}{\partial t}( st^2) \\
  &= \left(s\cdot\frac{\partial}{\partial t}t^2\right) +\left(t^2\cdot\frac{\partial}{\partial t}s\right) \\
  &= 2ts + 0 \\
  &= 2st\\
  \frac{\partial \phi}{\partial t}
  &= \frac{\partial}{\partial t}( s^2t )\\
  &= \left(s^2\cdot\frac{\partial}{\partial t}t\right) + \left(t\cdot\frac{\partial}{\partial t}s^2\right) \\
  &= s^2 + 0 \\
  &= s^2
\end{align*}

Al aplicar el caso 2 de la regla de la cadena, obtenemos
\begin{align*}
  \frac{\partial z}{\partial s}
  &= \frac{\partial z}{\partial \theta}\frac{\partial \theta}{\partial s} + \frac{\partial z}{\partial \phi}\frac{\partial \phi}{\partial s} \\
  &=  (\cos{\theta}\cos{\phi} \cdot t^2) + (-\sin{\theta}\sin{\phi} \cdot 2st) \\
  &= t^2\cos{\theta}\cos{\phi} - 2st\sin{\theta}\sin{\phi} \\
  \therefore \frac{\partial z}{\partial s}
  &= t^2\cos{\theta}\cos{\phi} - 2st\sin{\theta}\sin{\phi} \\ \\
  \frac{\partial z}{\partial t}
  &= \frac{\partial z}{\partial \theta}\frac{\partial \theta}{\partial t} + \frac{\partial z}{\partial \phi}\frac{\partial \phi}{\partial t} \\
  &=  (\cos{\theta}\cos{\phi} \cdot 2st) + (-\sin{\theta}\sin{\phi} \cdot s^2) \\
  &= 2st\cos{\theta}\cos{\phi} - s^2\sin{\theta}\sin{\phi} \\
  \therefore \frac{\partial z}{\partial t}
  &= 2st\cos{\theta}\cos{\phi} - s^2\sin{\theta}\sin{\phi} 
\end{align*}

% 6 -------------------------------------------------------------------------------------------------------------
\section{}

Sea $z = x^4 + x^2y$, con $x = s + 2t - u$, $y = stu^2$, utilice la \textbf{regla de la cadena} para calcular: $\frac{\partial z}{\partial s}$ $\frac{\partial z}{\partial t}$ $\frac{\partial z}{\partial u}$, donde $s=4$ $t=2$, $u=1$.

Primero calculamos las derivadas parciales de primer orden necesarias:

\begin{align*}
  \frac{\partial z}{\partial x}
  &= \frac{\partial}{\partial x}(x^4 + x^2y) \\
  &= 4x^3+2xy \\ \\
  \frac{\partial z}{\partial y}
  &= \frac{\partial}{\partial y}(x^4 + x^2y) \\
  &= x^2 
\end{align*}

Así, al aplicar el caso 2 de la regla de la cadena, obtenemos
\begin{align*}
  \frac{\partial z}{\partial s}
  &= \frac{\partial z}{\partial x}\frac{\partial x}{\partial s} + \frac{\partial z}{\partial y}\frac{\partial y}{\partial s} \\
  &= \left[(4x^3+2xy)\cdot \frac{\partial}{\partial s}(s + 2t - u)\right] +
  \left[x^2 \cdot \frac{\partial}{\partial s}(stu^2)\right] \\
  &= 4x^3+2xy + x^2tu^2
\end{align*}

\begin{align*}
  \frac{\partial z}{\partial t}
  &= \frac{\partial z}{\partial x}\frac{\partial x}{\partial t} + \frac{\partial z}{\partial y}\frac{\partial y}{\partial t} \\
  &= \left[(4x^3+2xy)\cdot \frac{\partial}{\partial t}(s + 2t - u)\right] +
  \left[x^2 \cdot \frac{\partial}{\partial t}(stu^2)\right] \\
  &= 2(4x^3+2xy) + x^2su^2 \\
  &= 8x^3 + 4xy + x^2su^2
\end{align*}

\begin{align*}
  \frac{\partial z}{\partial u}
  &= \frac{\partial z}{\partial x}\frac{\partial x}{\partial u} + \frac{\partial z}{\partial y}\frac{\partial y}{\partial u} \\
  &= \left[(4x^3+2xy)\cdot \frac{\partial}{\partial u}(s + 2t - u)\right] +
  \left[x^2 \cdot \frac{\partial}{\partial u}(stu^2)\right] \\
  &= -(4x^3+2xy) + 2x^2stu \\
  &= -4x^3 - 2xy + 2x^2stu
\end{align*}

Cuando $s=4$, $t=2$, y $u=1$, tenemos $x=4+2(2)-1=7$ y $y=(4)(2)(1^2)=8$, de modo que
\begin{align*}
  \frac{\partial z}{\partial s}
  &= 4x^3+2xy + x^2tu^2 \\
  &= 4(7^3)+2(7)(8)+(7^2)(2)(1^2) \\
  &= 1372 + 112 + 98 \\
  &= 1582 
\end{align*}
\begin{align*}
  \frac{\partial z}{\partial t}
  &= 8x^3 + 4xy + x^2su^2 \\
  &= 8(7^3) + 4(7)(8) + (7^2)(4)(1^2) \\
  &= 2744 + 224 + 196 \\
  &=3164
\end{align*}
\begin{align*}
  \frac{\partial z}{\partial u}
  &= -4x^3 - 2xy + 2x^2stu \\
  &= -4(7^3) - 2(7)(8) + 2(7^2)(4)(2)(1) \\
  &= -1372 - 112 + 784 \\
  &= -700
\end{align*}

$\therefore \frac{\partial z}{\partial s}=1582$, $\frac{\partial z}{\partial t} = 3164$ y $ \frac{\partial z}{\partial u} = -700$

% 7 -------------------------------------------------------------------------------------------------------------
\section{}

Sea $f(x, y, z) = x^2yz - xyz^3$, $P(2, -1, 1)$, $\hat{u} = \left (0,\frac{4}{5},\frac{-3}{5} \right)$:

\begin{itemize}[format=\textbf]
  
\item Determine el \textbf{gradiente} de la función escalar $f(x, y, z)$.

\item Evalúe el \textbf{gradiente} en el punto $P$.

\item Encuentre la \textit{razón de cambio} de $f(x, y, z)$ en el punto $P$ en la dirección del vector $\hat{u}$.
\end{itemize}

% 8 -------------------------------------------------------------------------------------------------------------
\section{}

Determine la máxima \textbf{razón de cambio} de $f(x, y) = 4y\sqrt{x}$ en el punto $P(4, 1)$ y la dirección en la cuál se presenta.

% 9 -------------------------------------------------------------------------------------------------------------
\section{}

Sea $f(x, y) = x^2 + xy + y^2 + y$. Calcule los valores \textbf{máximo} y \textbf{mínimo} \textit{locales}, y \textit{punto(s)} silla de la función.

% 10 -------------------------------------------------------------------------------------------------------------
\section{}

Sea $f(x, y) = x^2 + y^2 − 2x$, donde $D$ es la región triángular cerrada con vértices $A(2, O)$, $B(O, 2)$, $C(0, -2)$.

Determine los \textbf{valores máximos absolutos}, \textbf{valores mínimos absolutos} de $f(x, y)$ sobre el conjunto $D$.

% 11 -------------------------------------------------------------------------------------------------------------
\section{}

Encuentre tres números positivos cuya suma es $100$ y cuyo producto es un máximo.

% 12 -------------------------------------------------------------------------------------------------------------
\section{}

Utilizando \textbf{multiplicadores de Lagrange}, encuentre los valores \textbf{máximo} y \textbf{mínimo} de la función sujeta a la \textbf{restricción(es)} dadas.

\begin{itemize}[format=\textbf]

\item $f(x, y) = x^2 + y^2$, sujeto a la restricción $xy = 1$.

\item $f(x, y) = xyz$, sujeto a la restricción $x^2 + 2y^2 + 3z^2 = 6$.
  
\end{itemize}

\end{document}
