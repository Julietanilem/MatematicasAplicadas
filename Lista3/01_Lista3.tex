\documentclass[12pt]{article}

\usepackage[spanish]{babel}
\usepackage{hyperref}
\usepackage{graphicx}
\usepackage{listings}
\usepackage{color}
\usepackage{multicol}
\usepackage{amssymb}
\usepackage{enumitem}
\usepackage{here}
\usepackage{dsfont}
\usepackage{amsmath}
\usepackage{tipa}
\usepackage{float}
\spanishdecimal{.}

\title{Matemáticas para las Ciencias Aplicadas I}
\title{
	Tercera Lista de Problemas \\
	\textbf{Primera  Parte} \\
	\vspace{1ex}
	\large Matemáticas para las Ciencias Aplicadas I \\
	Facultad de Ciencias, UNAM}

\date{\today}

\author{Flores Morán Julieta Melina \\ Zarco Romero José Antonio}

\begin{document}

\maketitle

%% De la sección 3.2: ejercicios 38 y 57.
%% De la sección 3.3: ejercicios 54, 70, 76 y 83.
%% De la sección 3.4: ejercicios 23, 36, 45 y 47.
 %% De la sección 3.6: ejercicios 58, 63 y 65

%% 3.2 -------------------------------------------------------------------------------------------------------------------------------------------------------------------------------------------------------------------------------
\section{Sección 3.2 \\ Derivadas De Funciones Logarítmicas}
% 38 -------------------------------------------------------------------------------------------------------------
\subsection{Ejercicio 38} name \\

Encuentre $dy/dx$ usando diferenciación logarítmica.
\[
y=\frac{\sin{x}\cos{x}\tan^3{x}}{\sqrt{x}}
\]
\begin{equation*}
  \begin{split}
  \ln(x)
  &= \ln(\sin{x}\cos{x}\tan^3{x})-\ln(\sqrt{x}) \\
  &= \ln(\sin{x})+\ln(\cos{x})+\ln(\tan^3{x})-\ln \left(x^{\frac{1}{2}}\right) \\
  &= \ln(\sin{x})+\ln(\cos{x})+3\ln(\tan{x})-\frac{1}{2}\ln x \\
  \frac{1}{y}\frac{dy}{dy}
  &= \frac{\cos{x}}{\sin{x}} + \frac{-\sin{x}}{\cos{x}} + \frac{3\sec^2{x}}{\tan{x}} - \frac{1}{2x} \\
  \frac{dy}{dy}
  &= y \left[ \cot{x}-\tan{x} + \frac{3\sec^2{x}}{\tan{x}} - \frac{1}{2x} \right] \\
  \therefore \frac{dy}{dy}
  &= \frac{\sin{x}\cos{x}\tan^3{x}}{\sqrt{x}}
  \left[ \cot{x}-\tan{x} + \frac{3\sec^2{x}}{\tan{x}} - \frac{1}{2x} \right] \\
  \end{split}
\end{equation*}

% 57 -------------------------------------------------------------------------------------------------------------
\subsection{Ejercicio 57} name \\

Sea $p$ el número de paramecios en una solución nutritiva $t$ días después del inicio de un experimento, y supongamos que $p$ es definido implícitamente como una función de $t$ por la ecuación
\[
0 = \ln p + 0.83 − \ln(2.3 − 0.0046p) − 2.3t
\]
Utilice la diferenciación implícita para demostrar que la tasa de cambio de $p$ con respecto a $t$ satisface la ecuación
\[
\frac{dp}{dt} = 0.0046p(500 − p)
\]
\begin{equation*}
  \begin{split}
    \ln p - \ln(2.3 − 0.0046p)
    &= 2.3t -0.83 \\
    \frac{1}{p}\frac{dp}{dt}+\frac{0.0046}{2.3-0.0046p}\frac{dp}{dt}
    &= 2.3 \\
    \frac{dp}{dt}\left[\frac{1}{p}+\frac{0.0046}{2.3-0.0046p}\right]
    &= 2.3 \\
    \frac{dp}{dt}\left[\frac{2.3-0.0046p+ 0.0046p}{p(2.3-0.0046p)}\right]
    &= 2.3 \\
    \frac{dp}{dt}\left[\frac{2.3}{2.3p-0.0046p^2}\right]
    &= 2.3 \\
    \frac{dp}{dt}
    &= 2.3\left[\frac{2.3p-0.0046p^2}{2.3}\right] \\
    &= 2.3p-0.0046p^2 \\
    \therefore \frac{dp}{dt}
    &= 0.0046p(500 − p)
  \end{split}
\end{equation*}

%% 3.3 -----------------------------------------------------------------------------------------------------------------------------------------------------------------------------------------------------------------------------
\section{Sección 3.3 \\ Derivadas De Funciones Trigonométricas Exponenciales E Inversas} 
% 54 -------------------------------------------------------------------------------------------------------------
\subsection{Ejercicio 54} name \\
% 70 -------------------------------------------------------------------------------------------------------------
\subsection{Ejercicio 70} name \\
% 76 -------------------------------------------------------------------------------------------------------------
\subsection{Ejercicio 76} name \\
% 83 -------------------------------------------------------------------------------------------------------------
\subsection{Ejercicio 83} name \\

%% 3.4 -----------------------------------------------------------------------------------------------------------------------------------------------------------------------------------------------------------------------------
\section{Sección 3.4 \\ Tasas Relacionadas} 
% 23 -------------------------------------------------------------------------------------------------------------
\subsection{Ejercicio 23} name \\
% 36 -------------------------------------------------------------------------------------------------------------
\subsection{Ejercicio 36} name \\
% 45 -------------------------------------------------------------------------------------------------------------
\subsection{Ejercicio 45} name \\
% 47 -------------------------------------------------------------------------------------------------------------
\subsection{Ejercicio 47} name \\

%% 3.6 -----------------------------------------------------------------------------------------------------------------------------------------------------------------------------------------------------------------------------
\section{Sección 3.6 \\ La Regla De L'Hôpital; Formas Indeterminadas} 
% 58 -------------------------------------------------------------------------------------------------------------
\subsection{Ejercicio 58} name \\
% 63 -------------------------------------------------------------------------------------------------------------
\subsection{Ejercicio 63} name \\
% 65 -------------------------------------------------------------------------------------------------------------
\subsection{Ejercicio 65} name \\

\end{document}
