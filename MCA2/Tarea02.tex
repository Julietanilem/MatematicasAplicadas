\documentclass[12pt]{article}

\usepackage[spanish]{babel}

\usepackage{amsmath}
\usepackage{amssymb}

\usepackage{hyperref}
\usepackage{graphicx}
\usepackage{listings}
\usepackage{color}
\usepackage{multicol}
\usepackage{enumitem}
\usepackage{here}
\usepackage{dsfont}
\usepackage{tipa}
\usepackage{float}
\usepackage{dsfont} 
\spanishdecimal{.}

\title{Matemáticas para las Ciencias Aplicadas II}
\title{
	\textbf{Tarea 02} \\
	\vspace{1ex}
	\large Matemáticas para las Ciencias Aplicadas II \\
	Facultad de Ciencias, UNAM}
\date{\today}
\author{Flores Morán Julieta Melina \\ Zarco Romero José Antonio}

\begin{document}
\maketitle

% 1 -------------------------------------------------------------------------------------------------------------
\section{}
Proporcione el dominio de la función vectorial.
$$
⃗\vec{r(t)}
=
\frac{t-2}{t+2}\hat{i}
+
\sin{t}\hat{j}
+
\ln{9-t^2}\hat{k}
$$

% 2 -------------------------------------------------------------------------------------------------------------
\section{}
Sea
$$
⃗\vec{r(t)}
=
\frac{t^2-t}{t-1}\hat{i}
+
\sqrt{t+8}\hat{j}
+
\frac{\sin{\pi t}}{\ln{t}}\hat{k}
$$
Calcule $$ \lim{t \to 0} ⃗\vec{r(t)} $$

% 3 -------------------------------------------------------------------------------------------------------------
\section{}
Realice a mano la gráfica de las siguientes funciones vectoriales, indicando el sentido en que se traza la curva:
\begin{itemize}[format=\textbf]

\item 

\end{itemize}

% 4 -------------------------------------------------------------------------------------------------------------
\section{}
Proporcione las coordenadas del punto donde se intersecta la \textbf{hélice} $$, y la \textbf{esfera} $$.

% 5 -------------------------------------------------------------------------------------------------------------
\section{}
Dibuje las proyecciones de la curva
⃗$$
$$
sobre los planos $XY , XZ, YZ$. Utilice dichas proyecciones para hacer un esbozo de la curva.

% 6 -------------------------------------------------------------------------------------------------------------
\section{}
Las trayectorias de dos partículas están dadas por las siguientes funciones vectoriales:
\begin{itemize}[format=\textbf]

\item 

\end{itemize}
¿Chocarán las partículas? ¿En qué punto? ¿Se cortarán las trayectorias?

% 7 -------------------------------------------------------------------------------------------------------------
\section{}
Proporcione las coordenadas del punto sobre la curva

% 8 -------------------------------------------------------------------------------------------------------------
\section{}
Proporcione las coordenadas del punto donde se intersectan las curvas:
\begin{itemize}[format=\textbf]

\item 

\end{itemize}

% 9 -------------------------------------------------------------------------------------------------------------
\section{}
Determine la \textbf{longitud de curva} para las siguientes curvas:
\begin{itemize}[format=\textbf]

\item 

\end{itemize}

% 10 -------------------------------------------------------------------------------------------------------------
\section{}
Reparametrice la siguiente curva (respecto a la longitud de arco medida desde el punto donde $t = 0$), en la dirección en que $t$ se incrementa.
$$ = $$

\end{document}

