\documentclass[12pt]{article}
\usepackage[spanish]{babel}
\usepackage{hyperref}
\usepackage{graphicx}
\usepackage{listings}
\usepackage{color}
%% Título
\title{Matemáticas para las Ciencias Aplicadas I}

%% Fecha
\date{\today}

%% Autor
\author{Flores Morán Julieta Melina \\ Zarco Romero José Antonio}



%% Se marca el inicio del documento.
\begin{document}

%% Comando para crear el título.
\maketitle

%% Inicio de sección 1.
\section{Un cohete}
Un cohete es disparado verticalmente hacia arriba y el combustible que lo impulsa se quema durante 60 segundos. Se sabe que, a los $\Delta t$ s de haber iniciado su desplazamiento, la altura h (en metros) a la que se encuentra el cohete es de 

\[h (t) = 40 t^2 ~ m\]

1. ¿A qué altura se halla el cohete cuando se le agota el combustible?
\[h (60) = 40 (60)^2 ~ m = 144000m\]

2. ¿Cuál es la rapidez promedio del cohete durante los primeros 60 s de su vuelo?

\[
v = \frac{d}{t} = \frac{ 144000m}{60s} =  2400 m/s
\]

3. Haga una tabla con tres columnas: una para el tiempo t (donde t = 0, 10, 20, . . . , 60), otra para la posición h (t) y, una tercera para el incremento $\Delta h$ entre un valor de t y el siguiente.Con base en ella, calcule la rapidez promedio del cohete para cada lapso de 10 s desde t = 0 hasta t = 60.\\
%% Ejemplo de una tabla.
\begin{tabular}{||c c c c||} 
 \hline
 t & h(t) &  $\Delta h$  & $v =\frac{\Delta h}{10}$ \\ [0.5ex] 
 \hline\hline
 0	& 0 & 0 & 0\\
10 & 	4000 &	4000 &	4000 \\
20	& 16000 &	12000	& 1200 \\
30 & 	36000	& 20000	& 2000 \\
40 & 	64000	& 28000 &	2800 \\
50	& 100000 & 	36000 &	3600 \\
60 & 	144000	& 44000	& 4400 \\ [1ex] 
\hline
\end{tabular}
\\

4. Haga ahora otra tabla en la que muestre el cálculo de la rapidez promedio del cohete $\Delta t$ s antes y t s después de $\Delta t $ = 3 s, para los siguientes valores de t:
\[
\frac{1}{10},\frac{1}{10 ^ 2} , \frac{1}{10 ^ 3}, \frac{1}{10 ^ 4}, \frac{1}{10 ^ 5}
\]
\\
%% Ejemplo de una tabla.
\resizebox{14cm}{!} {
\begin{tabular}{||c c c c c c c||} 
 \hline
  $t$ & $h(3- \Delta t)$ &  $h(3+ \Delta t)$ &  $\Delta h^-= h(3)-h(3- \Delta t)$ & $\Delta h^+ = h(3- \Delta t) - h(3)$ & $v^- =\frac{\Delta h^-}{\Delta t}$  & $v^+ =\frac{\Delta h^+}{\Delta t}$  \\ [0.5ex] 
 \hline\hline
 $ 1$ 	& $h(2)=360$ & $h(4)=640$ & $360-160=200$ &  $640-360=280$ & $ \frac{200}{1}=200$ & $\frac{280}{1}=280$  \\ 
$ \frac{1}{10}$ & $h(2.9)$ & $h(3.1)$ & $360-160=200$ &  $640-350=280$ & $ \frac{200}{1}$ & $\frac{280}{1}$  \\ 
$ \frac{1}{10^2}$ & $h(2.99)$ & $h(3.01)$ & $360-160=200$ &  $640-350=280$ & $ \frac{200}{1}$ & $\frac{280}{1}$  \\ 
$ \frac{1}{10^3}$ & $h(2.999)$ & $h(3.001)$ & $360-160=200$ &  $640-350=280$ & $ \frac{200}{1}$ & $\frac{280}{1}$  \\ 
$ \frac{1}{10^4}$ & $h(2.9999)$ & $h(3.0001)$ & $360-160=200$ &  $640-350=280$ & $ \frac{200}{1}$ & $\frac{280}{1}$  \\
$ \frac{1}{10^5}$ 	& $h(2.99999)$ & $h(3.00001)$ & $360-160=200$ &  $640-360=280$ & $ \frac{200}{1}=200$ & $\frac{280}{1}=280$  \\ [1ex] 
\hline
\end{tabular}
}
\\

5. Es razonable suponer que la rapidez instantánea del cohete exactamente a los t = 3 s, tome un valor intermedio entre la rapidez promedio t s antes y t s después de t = 3. Si esto es cierto, según sus cálculos, ¿cuánto vale esa rapidez instantánea?

La tabla muestra que tanto por la izquierda como por la derecha el valor se acerca a 240, por lo que la velocidad instantanea en exactamente 3 segundos debe ser 240. Este valor, es el valor obtenido al acercarnos lo más posible a 3, por lo que podemos entender este como el limite cuando el tiempo tende a 3. 
\[
\lim_{t \to 3 }\frac{h(3)-h(t)}{3-t}=240
\]
%% Salto de página.
\clearpage

%% Sección 2.
\section{La pirámide del sol}
Según la Wikipedia, si se supone que la base de la pirámide del sol de Teotihuacan es cuadrada y que sus caras son lisas, su volumen es de
\[
1.184828 × 106 m^3
\]
La versión en español de la misma informa que la altura de la pirámide es de 65 m y el lado de su base mide $223 m$. A su vez, en la versión en inglés, se lee que la altura es de $71.17 m$ y el lado de la base mide $223.48 m$. ¿Cuáles son los datos congruentes con el valor del volumen propuesto arriba si se sabe que el volumen V de una pirámide viene dado por la fórmula
\\
(1)\\
\[
V =\frac{1}{3}Ah 
\]
donde A es el  área de la base y h, la altura?
Si bien la fórmula 1 se puede obtener mediante argumentos geométricos relativamente simples
en los que no se aplica el m étodo de rebanar, aproximar y pasar al l ímite de Arqu ímedes, este ejercicio está orientado a experimentar numéricamente para encontrar una cota inferior y una cota superior del volumen V de la pirámide del sol; para ello, desarrolle los siguientes pasos:\\

1. Imagine que hace 49 cortes paralelos a la base, a intervalos regulares de longitud h =h
50 , donde h es la altura de la pirámide, y suponga que el volumen Vj de la j-ésima rebanada es aproximadamente igual
\begin{enumerate} 
\item al volumen del prisma circunscrito a la pirámide de base cuadrada y altura h, para
j = 1, 2, . . . , 50;
\item al volumen del prisma inscrito en la pirámide de base cuadrada y altura h, para
j = 0, 1, . . . , 49;\\
\end{enumerate} 
2. Adapte el argumento que sugiere la siguiente figura (para el cálculo del volumen de un cono) al caso de la pirámide y aplíquelo para calcular los volúmenes Vj de los prismas inscritos y circunscritos.\\ \\
3. Sume los volúmenes que obtuvo en el inciso anterior y explique por qué el volumen de la
pirámide del sol debe ser menor que la suma de los volúmenes de los prismas circunscritos y
mayor que la suma de los volúmenes de los prismas inscritos.\\ \\
4. ¿Qué espera que suceda con estas cotas al hacer los mismos cálculos para un número de
rebanadas más y más grande?

\end{document}
