\documentclass[12pt]{exam}
\usepackage[utf8]{inputenc}		% Caracteres latinos
\usepackage[spanish]{babel}		% Idioma español
\usepackage{geometry}			% Organizar el documento
\usepackage{graphicx}			% Incluir gráficos
\usepackage{makecell}			% Para personalizar las celdas de una tabla
\usepackage[nohdr]{mathexam}	% Añadimos el paquete mathexam (sin header)
\usepackage{amsmath}
\usepackage{amsfonts}
\usepackage{amssymb}
\usepackage{mathtools}
\usepackage{tikz}
\usepackage{pgfplots}
\pgfplotsset{compat=1.10}
\usepgfplotslibrary{fillbetween}
%\usetikzlibrary{positioning}    % yo
\usepgfplotslibrary{polar}
\usepackage[shortlabels]{enumitem}
\renewcommand{\baselinestretch}{1.5}
\usepackage{mathtools}
\usepackage{bm}
\usepackage{esvect}
\usepackage[fleqn]{mathtools}
\usepackage{relsize}
\usepackage{multirow}
\usepackage{multicol}
\usepackage[document]{ragged2e}
\usepackage{textpos}
\usepackage{tcolorbox}
\usepackage{hyperref}
\usepackage{enumerate}
%% \usepackage{wrapfig} % wrapfigure
%% \usepackage{float}
%% \usepackage{graphicx}
\usepackage{here} % [H]
\usepackage{soul}
\spanishdecimal{.}


\geometry{
  a4paper,                    % Tamaño del documento
  hmargin = {1.7cm, 1.7cm}, 	% Margen horizontal izquierdo, derecho
  vmargin = {1cm, 1cm},	    % Margen vertical superior, inferior
  headsep = 4mm,				% Separación entre el encabezado y el texto
  head = .2cm,				% Tamaño del encabezado
  % marginparsep = 5mm, 		% Seperación entre las notas y el texto
  % marginpar = 1.5cm,		% Tamaño de las notas
  includeall,                 % incluye el encabezado, footer y notas dentro del tamaño del documento
  nomarginpar,	            % Elimina las notas
  foot = 1cm,                 % Tamaño del footer
  twoside,                	% Habilita el modo de impresión a doble cara
}

\selectlanguage{spanish}       
\spanishdecimal{.}

\newcommand{\iuni}{\pmb{\hat{\imath}}}
\newcommand{\juni}{\pmb{\hat{\jmath}}}
\newcommand{\kuni}{\pmb{\hat{k}}}
% DOCUMENTO


\begin{document}

\centering

\Large 
\textbf{Proyecto 2}\\
\large 
Unidad 2: Integrales Triples. Integrales de línea. \\
Alumno: Zarco Romero José Antonio\\
Valor: 4 puntos\\
\normalsize
Fecha de entrega: 
Viernes 11/10/2024

\vskip10pt

\normalsize

\pointpoints{punto}{puntos}
\pointformat{\bfseries\boldmath(\thepoints)}
\vskip10pt

\begin{tcolorbox}
  \textbf{Instrucciones:}Resuelve con detalle el siguiente problema, describiendo cada uno de los pasos de
  tu análisis. Puedes entregarlo en físico el viernes o subir tu archivo al aula virtual a mas tardar a
  las 23:59 hrs.
\end{tcolorbox}

\begin{questions}

  \question{Usa coordenadas cilíndricas para mostrar que el volumen del sólido delimitado
    superiormente por la esfera $r^2 + z^2 = a^2$ e inferiormente por el cono $z = r\cot{(\phi_0)}$ (o
    $\phi = \phi_0$), donde $0 < \phi_0 < \pi/2$, es}

  \begin{align*}
    V = \frac{2 \pi a^3}{3} (1 - \cos{(\phi_0)}) && (1)
  \end{align*}

  De la ecuación de la esfera $r^2 + z^2 = a^2$, tenemos que $z^2= a^2 - r^2$, igualando con $z^2 = (r\cot{(\phi_0)})^2$ obtenemos

  \begin{align*}
    (r\cot{(\phi_0)})^2 &=  a^2 - r^2\\
    (r\cot{(\phi_0)})^2 + r^2 &= a^2 \\
    r^2(\cot^2{(\phi_0)} + 1) &= a^2 \\
    r^2 (\csc^2{(\phi_0)}) &= a^2 \\
    r^2 &= \frac{a^2}{\csc^2{(\phi_0)}} \\
    r &= \sqrt{\frac{a^2}{\csc^2{(\phi_0)}}} \\
    r &= \frac{a}{\csc{(\phi_0)}} \\
    r &= a \sin{(\phi_0)}
  \end{align*}

  De esta forma, tenemos que $r$ va de 0 a $a\sin{(\phi_0)}$

  De la misma ecuación de la esfera $r^2 + z^2 = a^2$, tenemos que $z^2= a^2 - r^2$, o bien $z= \sqrt{a^2 - r^2}$. Por lo que,
  $z$ va de $r\cot{(\phi_0)}$ a $\sqrt{a^2 - r^2}$

  Así, tenemos que la región de integración descrita en coordenadas cilíndricas es

  \[
  S = \left\{(r,\theta, z)~|~0 \leq \theta \leq 2\pi,~0\leq r \leq a\sin{(\phi_0)},~r\cot{(\phi_0)} \leq z \leq \sqrt{a^2 - r^2} \right\}
  \]

  Luego, el volumen del solido es

  \begin{align*}
    V &= \iiint_S dV \\
    &= \int_0^{2\pi} \int_0^{a\sin{(\phi_0)}} \int_{r\cot{(\phi_0)}}^{\sqrt{a^2 - r^2}} r ~ dz ~ dr ~ d\theta \\
    &= \int_0^{2\pi} ~ d\theta \cdot \int_0^{a\sin{(\phi_0)}} r \left[\int_{r\cot{(\phi_0)}}^{\sqrt{a^2 - r^2}}~ dz\right]~dr \\
    &= 2\pi \int_0^{a\sin{(\phi_0)}} r [(\sqrt{a^2 - r^2}) - (r\cot{(\phi_0)})] ~ dr \\
    &= 2\pi \int_0^{a\sin{(\phi_0)}} r\sqrt{a^2 - r^2} - r^2 \cot{(\phi_0)} ~ dr \\
    &= 2\pi \left[\int_0^{a\sin{(\phi_0)}} r\sqrt{a^2 - r^2}~dr - \int_0^{a\sin{(\phi_0)}}r^2 \cot{(\phi_0)} ~ dr \right]
  \end{align*}

  Sea $u = a^2-r^2$, entonces $du=-2r~dr$ y $r~dr= -\frac{du}{2}$, además $u(a\sin{(\phi_0)}) = a^2 - a^2\sin^2{(\phi_0)} = 
  a^2(1-\sin^2{(\phi_0)}) = a^2\cos^2{(\phi_0)}$ y $u(0) = a^2$. Entonces, el volumen del solido es

  \begin{align*}
    V 
    &= 2\pi \left[-\frac{1}{2}\int_{a^2}^{a^2\cos^2{(\phi_0)}}  u^{1/2} ~du - \int_0^{a\sin{(\phi_0)}}r^2 \cot{(\phi_0)} ~ dr \right]\\
    &= 2\pi \left\{-\frac{1}{2} \left[\frac{u^{3/2}}{3/2}\right]_{u=a^2}^{u=a^2\cos^2{(\phi_0)}}- \int_0^{a\sin{(\phi_0)}}r^2 \cot{(\phi_0)} ~ dr \right\}\\
    &= 2\pi \left\{-\frac{1}{2} \left[\frac{u^{3/2}}{3/2}\right]_{u=a^2}^{u=a^2\cos^2{(\phi_0)}}- \frac{1}{3}\left[r^3 \cot{(\phi_0)}\right]_{r=0}^{r=a\sin{(\phi_0)}} \right\}\\
    &= 2\pi \left\{-\frac{1}{2} \cdot \frac{2}{3}\left[u^{3/2}\right]_{u=a^2}^{u=a^2\cos^2{(\phi_0)}}- \frac{1}{3}\left[r^3 \cot{(\phi_0)}\right]_{r=0}^{r=a\sin{(\phi_0)}} \right\}\\
    &= -\frac{2\pi}{3} \left\{\left[u^{3/2}\right]_{u=a^2}^{u=a^2\cos^2{(\phi_0)}} + \left[r^3 \cot{(\phi_0)}\right]_{r=0}^{r=a\sin{(\phi_0)}} \right\}\\
    &= -\frac{2\pi}{3} \left\{\left[u^{3/2}\right]_{u=a^2}^{u=a^2\cos^2{(\phi_0)}} + a^3\sin^3{(\phi_0)} \cot{(\phi_0)} \right\}\\
    &= -\frac{2\pi}{3} \left\{(a^3\cos^3{\phi_0}-a^3) + a^3\sin^3{(\phi_0)} \cot{(\phi_0)} \right\}\\
    &= -\frac{2\pi a^3}{3} \left\{\cos^3{\phi_0}-1 + \sin^3{(\phi_0)} \cot{(\phi_0)} \right\}\\
    &= -\frac{2\pi a^3}{3} \left\{\cos^3{(\phi_0)}-1 + \sin^3{(\phi_0)}\cdot\frac{\cos{(\phi_0)}}{\sin{(\phi_0)}} \right\}
  \end{align*}
  \begin{align*}
    &= -\frac{2\pi a^3}{3} \left\{\cos^3{(\phi_0)}-1 + \sin^2{(\phi_0)}\cos{(\phi_0)} \right\}\\
    &= -\frac{2\pi a^3}{3} \left\{-1+\cos{(\phi_0)}[\cos^2{(\phi_0)} +\sin^2{(\phi_0)}]\right\}\\
    &= -\frac{2\pi a^3}{3} \left\{-1+\cos{(\phi_0)}\right\}\\
    &= \frac{2 \pi a^3}{3} (1 - \cos{(\phi_0)})
  \end{align*}

  $\therefore V = \frac{2 \pi a^3}{3} (1 - \cos{(\phi_0)})$

  %2
  \question{Deduce que el volumen de la cuña esférica dada por $\rho_1 \leq \rho \leq \rho_2$, $\theta_1 \leq \theta \leq \theta_2$, $\phi_1 \leq \phi \leq \phi_2$
  es}

  \begin{align*}
    \Delta V = \frac{\rho_2^3 - \rho_1^3}{3} (\cos{(\phi_1)} - \cos{(\phi_2)})(\theta_2 - \theta_1) && (2)
  \end{align*}

  Tenemos que la región de integración puede ser descrita en coordenadas esféricas como

  \[
  E = \left\{ (\rho, \theta, \phi) ~ | ~ \rho_1 \leq \rho \leq \rho_2, \theta_1 \leq \theta \leq \theta_2, \phi_1 \leq \phi \leq \phi_2 \right\}
  \]

  Entonces, el volumen de la cuña esférica es

  \begin{align*}
    \int_{\phi_1}^{\phi_2} \int_{\theta_1}^{\theta_2} \int_{\rho_1}^{\rho_2} \rho^2 \sin{\phi} \, d\rho \, d\theta \, d\phi 
    &= \int_{\phi_1}^{\phi_2} \sin{\phi} ~d\phi ~ \int_{\theta_1}^{\theta_2} ~d\theta ~ \int_{\rho_1}^{\rho_2} \rho^2 ~d\rho \\
    &= -\cos{\phi}\Big|_{\phi= \phi_1}^{\phi = \phi_2} \cdot (\theta_2 - \theta_1) \cdot \frac{1}{3} \rho^3\Big|_{\rho = \rho_1}^{\rho = \rho_2}\\
    &= [(-\cos{\phi_2}) - (-\cos{\phi_1})] \cdot (\theta_2 - \theta_1) \cdot \frac{1}{3} (\rho_2^3 - \rho_1^3)\\
    &= \frac{\rho_2^3 - \rho_1^3}{3} (\cos{(\phi_1)} - \cos{(\phi_2)}) (\theta_2 - \theta_1)
  \end{align*}

  $\therefore \Delta V = \frac{\rho_2^3 - \rho_1^3}{3} (\cos{(\phi_1)} - \cos{(\phi_2)}) (\theta_2 - \theta_1)$

  %3
  \question{Usa el Teorema del Valor Medio para probar que el volumen del inciso (b) puede ser escrito
    como}

  \begin{align*}
    \Delta V = \tilde{\rho}^2 \sin(\tilde{\phi}) \Delta \rho \Delta \theta \Delta \phi && (3)
  \end{align*}

  donde $\tilde{\rho}$ se encuentra entre $\rho_1$ y $\rho_2, \tilde{\phi}$ se encuentra entre $\phi_1$ y $\phi_2$, $\Delta \rho = \rho_2 - \rho_1$, $\Delta \theta = \theta_2 - \theta_1$, y $\Delta \phi = \phi_2 - \phi_1$.

  Para la integral $\int_{\rho_1}^{\rho_2} \rho^2 ~ d\rho$ aplicamos el Teorema del Valor Medio. 
  Sabemos que existe un $\tilde{\rho}$ tal que $\rho_1 \leq \tilde{\rho} \leq \rho_2$. Entonces,

  \begin{align*}
    \int_{\rho_1}^{\rho_2} \rho^2 ~ d\rho 
    &= \tilde{\rho}^2 (\rho_2 - \rho_1) \\
    &= \tilde{\rho}^2 \cdot \Delta \rho
  \end{align*}

  Igualmente, para la integral $\int_{\phi_1}^{\phi_2} \sin{\phi} \, d\phi$ aplicamos el Teorema del Valor Medio.
  Sabemos que existe un $\tilde{\phi}$ tal que $\phi_1 \leq \tilde{\phi} \leq \phi_2$. Entonces,

  \begin{align*}
    \int_{\phi_1}^{\phi_2} \sin{\phi} \, d\phi
    &= \sin{\tilde{\phi}} (\phi_2 - \phi_1) \\
    &= \sin{\tilde{\phi}} \cdot \Delta \phi
  \end{align*}

  Además, para la integral $\int_{\theta_1}^{\theta_2} \theta \, d\theta$, ya que el integrando es una constante, el resultado es 

  \begin{align*}
    \int_{\theta_1}^{\theta_2}  \, d\theta
    &= \theta_2 - \theta_1 \\
    &= \Delta \theta
  \end{align*}

  Juntando los resultados de las integrales anteriores, el volumen del inciso (b) puede ser escrito como

  \begin{align*}
    \int_{\phi_1}^{\phi_2} \int_{\theta_1}^{\theta_2} \int_{\rho_1}^{\rho_2} \rho^2 \sin{\phi} \, d\rho \, d\theta \, d\phi 
    &= \int_{\phi_1}^{\phi_2} \sin{\phi} ~d\phi ~ \int_{\theta_1}^{\theta_2} ~d\theta ~ \int_{\rho_1}^{\rho_2} \rho^2 ~d\rho \\
    &= (\sin{\tilde{\phi}} \cdot \Delta \phi) \cdot (\Delta \theta) \cdot (\tilde{\rho}^2 \cdot \Delta \rho) \\
    &= \tilde{\rho}^2 \sin{\tilde{\phi}} \Delta \rho \Delta \theta \Delta \phi
  \end{align*}

  $\therefore \Delta V = \tilde{\rho}^2 \sin{\tilde{\phi}} \Delta \rho \Delta \theta \Delta \phi$

\end{questions}

\end{document} 
