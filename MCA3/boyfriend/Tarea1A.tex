\documentclass[12pt]{exam}
\usepackage[utf8]{inputenc}		% Caracteres latinos
\usepackage[spanish]{babel}		% Idioma español
\usepackage{geometry}			% Organizar el documento
\usepackage{graphicx}			% Incluir gráficos
\usepackage{makecell}			% Para personalizar las celdas de una tabla
\usepackage[nohdr]{mathexam}	% Añadimos el paquete mathexam (sin header)
\usepackage{amsmath}
\usepackage{amsfonts}
\usepackage{amssymb}
\usepackage{mathtools}
\usepackage{tikz}
\usepackage{pgfplots}
\pgfplotsset{compat=1.10}
\usepgfplotslibrary{fillbetween}
%\usetikzlibrary{positioning}    % yo
\usepgfplotslibrary{polar}
\usepackage[shortlabels]{enumitem}
\renewcommand{\baselinestretch}{1.5}
\usepackage{mathtools}
\usepackage{bm}
\usepackage{esvect}
\usepackage[fleqn]{mathtools}
\usepackage{relsize}
\usepackage{multirow}
\usepackage{multicol}
\usepackage[document]{ragged2e}
\usepackage{textpos}
\usepackage{tcolorbox}
\usepackage{hyperref}
%% \usepackage{wrapfig} % wrapfigure
%% \usepackage{float}
%% \usepackage{graphicx}
\usepackage{here} % [H]
\spanishdecimal{.}


\geometry{
  a4paper,                    % Tamaño del documento
  hmargin = {1.7cm, 1.7cm}, 	% Margen horizontal izquierdo, derecho
  vmargin = {1cm, 1cm},	    % Margen vertical superior, inferior
  headsep = 4mm,				% Separación entre el encabezado y el texto
  head = .2cm,				% Tamaño del encabezado
  % marginparsep = 5mm, 		% Seperación entre las notas y el texto
  % marginpar = 1.5cm,		% Tamaño de las notas
  includeall,                 % incluye el encabezado, footer y notas dentro del tamaño del documento
  nomarginpar,	            % Elimina las notas
  foot = 1cm,                 % Tamaño del footer
  twoside,                	% Habilita el modo de impresión a doble cara
}

\selectlanguage{spanish}       
\spanishdecimal{.}


\newcommand{\iuni}{\pmb{\hat{\imath}}}
\newcommand{\juni}{\pmb{\hat{\jmath}}}
\newcommand{\kuni}{\pmb{\hat{k}}}
% DOCUMENTO
\begin{document}

\centering


\Large 
\textbf{Tarea A}\\
\large 
Unidad 1: Integral de Riemann. Integrales dobles.\\
Alumno: Zarco Romero José Antonio\\
Valor: 7 puntos\\
\normalsize
Fecha de entrega: 

Viernes 23/08/2024 durante la clase

\vskip10pt

\normalsize

\pointpoints{punto}{puntos}
\pointformat{\bfseries\boldmath(\thepoints)}
\vskip10pt

\begin{questions}

  % ------------------------------------------------------------------------------------------------------------------------------------------------------------------------------------------------------------------------------
  \question
  
  \begin{multicols}{2}

    Usa una suma de Riemann con $m=n=2$ para estimar el valor de $\mathlarger{\iint}_R\text{sen}(x+y)dA$, donde $R=[0,\pi]\times[0,\pi]$. Elige los puntos muestra como las esquinas inferiores izquierdas.

    El valor de la integral doble de $f(x, y) = \sin{(x+y)}$ sobre el rectángulo $R=[0, \pi] \times [0,\pi]$, utilizando una \textbf{doble suma de Riemann} con $m=n=2$, es:

    \begin{figure}[H]
      \centering
      \includegraphics[width=0.45\textwidth]{{./img/t1a_1.png}}
      \label{fig:sen1}
      \caption{$\sin{(x+y)}$}
    \end{figure}
  \end{multicols}
  $$    \int \int _R f(x,y) dA  \approx \sum_{i=1}^{2}\sum_{j=1}^{2} f(x_{i-1},y_{j-1})\Delta A$$
  
  donde $x_{i-1}$ y $y_{j-1}$ son las \textit{esquinas inferiores izquierdas}. Además, $\Delta A = \Delta x \cdot \Delta y = \frac{\pi-0}{2} \cdot \frac{\pi-0}{2}=\frac{\pi}{2}\cdot\frac{\pi}{2} = \frac{\pi^2}{4}$

  \begin{figure}[H]
    \centering
    \begin{tikzpicture}
      % Dominio
      \draw[orange, very thick, fill=orange!6] (0,0) rectangle (4,4);
      \draw[orange, very thick] (2,0) -- (2,4);
      \draw[orange, very thick] (0,2) -- (4,2);

      % Ejes
      \draw[thick] (-1,0) -- (4.5,0) node[right] {$x$};
      \draw[thick] (0,-1) -- (0,4.5) node[above] {$y$};

      % Coordenadas
      \node at (0.6,0.4) {$(0, 0)$};
      \node at (0.6,2.4) {$\left(0, \frac{\pi}{2}\right)$};
      \node at (2.6,2.4) {$\left(\frac{\pi}{2}, \frac{\pi}{2}\right)$};
      \node at (2.6,0.4) {$\left(\frac{\pi}{2}, 0\right)$};

      % Regiones
      \node at (1,1) {$R_{11}$};
      \node at (1,3) {$R_{12}$};
      \node at (3,1) {$R_{21}$};
      \node at (3,3) {$R_{22}$};

      % Puntos
      \filldraw[black] (0,0) circle (2pt);
      \filldraw[black] (2,0) circle (2pt);
      \filldraw[black] (0,2) circle (2pt);
      \filldraw[black] (2,2) circle (2pt);
      
      % Escala
      \node at (-0.5,-0.5) {0};
      \node at (-0.5, 2) {$\frac{\pi}{2}$};
      \node at (-0.5,4) {$\pi$};
      \node at (2,-0.5) {$\frac{\pi}{2}$};
      \node at (4,-0.5) {$\pi$};
    \end{tikzpicture}
    \label{fig:rec1}
    \caption{Rectángulo $R$ de integración}
  \end{figure}

  Entonces,

  \begin{align*}
    \int \int _R \sin{(x+y)} dA
    &\approx \sum_{i=1}^{2}\sum_{j=1}^{2} \left[ f(x_{i-1},y_{j-1}) \cdot  \frac{\pi^2}{4} \right] \\
    &= \frac{\pi^2}{4} \left[ \sum_{i=1}^{2}\sum_{j=1}^{2} f(x_{i-1},y_{j-1}) \right] \\
    &= \frac{\pi^2}{4} \left[f(0,0)+f\left(0,\frac{\pi}{2}\right)+f\left(\frac{\pi}{2},0\right)+f\left(\frac{\pi}{2},\frac{\pi}{2}\right) \right] \\
    &= \frac{\pi^2}{4} \left[\sin{(0+0)}+\sin{\left(0+\frac{\pi}{2}\right)}+\sin{\left(\frac{\pi}{2}+0\right)}+\sin{\left(\frac{\pi}{2}+\frac{\pi}{2}\right)} \right] \\
    &= \frac{\pi^2}{4} \left[\sin{(0)}+\sin{\left(\frac{\pi}{2}\right)}+\sin{\left(\frac{\pi}{2}\right)}+\sin{(\pi)} \right] \\
    &= \frac{\pi^2}{4} \left(0+1+1+0 \right) \\
    &= \frac{\pi^2}{4} \cdot 2 \\
    &= \frac{\pi^2}{2} 
  \end{align*}
  Éste es el valor estimado de $f(x, y) = \sin{(x+y)}$

  % ------------------------------------------------------------------------------------------------------------------------------------------------------------------------------------------------------------------------------
  \question
  En la siguiente figura se muestra un mapa de curvas de nivel para la función $f(x,y)$. Usa la Regla del Punto Medio con $m=n=2$ para estimar el valor de $\mathlarger{\iint}_Rf(x,y)dA$ en la región $R=[0,4]\times[0,4]$. 
  \begin{figure}[H]
    \centering
    \includegraphics[width=0.4\textwidth]{{./img/t1a_2.png}}
    \label{fig:curvas}
    \caption{Curvas de nivel de $f(x,y)$}
  \end{figure}
  Utilizando la \textbf{Regla del punto medio para integrales dobles} donde
  $$ \iint\limits_{\underset{\textstyle R}{}} f(x,y)\, dA\approx \sum_{i=1}^{m} \sum_{j=1}^{n} f\left(\overline{x}_i, \overline{y}_j\right) \Delta A$$
  donde \( \overline{x}_i \) es el punto medio de \([x_{i-1}, x_i]\) y \( \overline{y}_j \) es el punto medio de \([y_{j-1}, y_j]\).

  El área de cada subrectángulo es $\Delta A=\Delta x \cdot\Delta y= \frac{4-0}{2}\cdot\frac{4-0}{2}=2\cdot 2=4$. Así que, al usar el \hyperref[fig:curvas]{mapa de contorno} para estimar el valor de $f$ en el centro de cada subrectángulo, obtenemos
  \begin{align*}
    \mathlarger{\iint}_Rf(x,y)dA
    &\approx \sum_{i=1}^{2} \sum_{j=1}^{2} \left[f\left(\overline{x}_i, \overline{y}_j\right) \cdot 4\right]\\
    &= 4 \left[\sum_{i=1}^{2} \sum_{j=1}^{2}f\left(\overline{x}_i, \overline{y}_j\right) \right]\\
    &= 4 \left[f(1,1)+f(1,3)+f(3,1)+f(3,3) \right]\\
    &= 4 \left(27+4+14+17\right) \\
    &= 4\cdot 62\\
    &= 248
  \end{align*}
  Por tanto, se tiene $\mathlarger{\iint}_Rf(x,y)dA \approx 248$

  % ------------------------------------------------------------------------------------------------------------------------------------------------------------------------------------------------------------------------------
  \question
  Calcula las siguientes integrales iteradas.
  \begin{enumerate}[a)]
  \item $\mathlarger{\int}_0^2\mathlarger{\int}_0^{1}(2x+y)^8\,dx\,dy$
    Si se considera a $y$ como una constante, se obtiene $ \int_{0}^{1} (2x+y)^8 dx $. Si hacemos que $u = 2x+y$, entonces
    $$ \frac{du}{dx}=2 \qquad \text{ así, } \qquad \frac{du}{2}=dx $$
    Además, los límites de integración quedan expresados como $u(0)=2(0)+y=y$ y $u(1)=2(1)+y=2+y$. De este modo,
    \begin{align*}
      \int_{0}^{1} (2x+y)^8 dx 
      &= \int_{y}^{2+y} u^8 \frac{du}{2}  \\
      &= \frac{1}{2} \int_{y}^{2+y} u^8 du \\
      &= \frac{1}{2} \left[ \frac{u^{8+1}}{8+1}\right]_y^{2+y} \\
      &= \frac{1}{2} \left[ \frac{u^9}{9}\right]_y^{2+y}  \\
      &= \frac{1}{2} \left\{ \left[ \frac{(2+y)^9}{9}\right] - \left[ \frac{y^9}{9}\right] \right\} \\
      &= \frac{1}{18} \left[ (2+y)^9 - y^9 \right] 
    \end{align*}
    Si ahora se integra la función respecto a $y$, se obtiene:
    \begin{align*}
      \int_{0}^{2}\int_{0}^{1} (2x+y)^8 dx dy
      &= \int_{0}^{2} \left[\int_{0}^{1} (2x+y)^8 dx \right] dy \\
      &= \int_{0}^{2} \left\{ \frac{1}{18} \left[ (2+y)^9 - y^9 \right] \right\} dy \\
      &= \frac{1}{18} \int_{0}^{2} \left[ (2+y)^9 - y^9 \right] dy \\
      &= \frac{1}{18} \left[ \int_0^2(2+y)^9dy - \int_0^2y^9dy \right]
    \end{align*}
    Resolviendo ambas integrales por separado, se tiene que:
    
    \begin{multicols}{2}
      \begin{itemize}[format=\textbf]
      \item $\int_0^2(2+y)^9dy$
        
        Si hacemos que $u=2+y$, entonces
        $$ \frac{du}{dy}=1 \qquad \text{ así, } \qquad du=dy $$
        Además, los límites de integración quedan expresados como $u(0)=2+(0)=2$ y $u(2)=2+(2)=4$. De este modo,
        \begin{align*}
          \int_0^2(2+y)^9dy
          &= \int_{2}^{4} u^9 du  \\
          &= \left[\frac{u^{9+1}}{9+1}\right]_{2}^{4}\\
          &= \left[\frac{u^{10}}{10}\right]_{2}^{4}\\
          &= \frac{4^{10}}{10} - \frac{2^{10}}{10} \\
          &= \frac{(2^2)^{10}-2^{10}}{10}\\
          &= \frac{2^{20}-2^{10}}{10}
        \end{align*}
      \item $\int_0^2y^9dy$
        \begin{align*}
          \int_0^2y^9dy
          &= \left[\frac{y^{9+1}}{9+1}\right]_0^2\\
          &= \left[\frac{y^{10}}{10}\right]_0^2\\
          &= \frac{(2)^{10}}{10} - \frac{(0)^{10}}{10} \\
          &= \frac{2^{10}}{10} - \frac{0}{10}\\
          &= \frac{2^{10}}{10} - 0\\
          &= \frac{2^{10}}{10}
        \end{align*}
      \end{itemize}
    \end{multicols}
    
    Luego, sustituyendo los valores de las integrales, se tiene que:
    \begin{align*}
      \frac{1}{18} \left[ \int_0^2(2+y)^9dy - \int_0^2y^9dy \right] 
      &= \frac{1}{18} \left[\left(\frac{2^{20}-2^{10}}{10}\right)-\left(\frac{2^{10}}{10}\right)\right]\\
      &= \frac{1}{18} \left(\frac{2^{20}-2^{10}-2^{10}}{10}\right) \\
      &= \frac{1}{18} \left[\frac{2^{20}-2(2^{10})}{10}\right] \\
      &= \frac{1}{18} \left(\frac{2^{20}-2^{11}}{10}\right) 
    \end{align*}
    \begin{align*}
      &= \frac{1}{18} \left[\frac{2(2^{19}-2^{10})}{10}\right] \\
      &= \frac{1}{18} \left(\frac{2^{19}-2^{10}}{5}\right) \\
      &= \frac{1}{18} \left[\frac{2(2^{18}-2^{9})}{5}\right] \\
      &= \frac{2}{18} \left(\frac{2^{18}-2^{9}}{5}\right) \\
      &= \frac{1}{9} \left(\frac{2^{18}-2^{9}}{5}\right) \\
      &= \frac{2^{18}-2^{9}}{45} \\
      &= \frac{2^{9}}{45} (2^9-1) \\
      &= \frac{261632}{45}
    \end{align*}
    $\therefore \mathlarger{\int}_0^2\mathlarger{\int}_0^{1}(2x+y)^8\,dx\,dy = \frac{2^{9}}{45} (2^9-1) \approx 5814.0444$
    \begin{figure}[H]
      \centering
      \includegraphics[scale=0.4]{{./img/t1a_3.png}}
      \label{fig:3a}
      \caption{$(2x+y)^8$}
    \end{figure}
    % --------------------------------------------------------------------------------
  \item $\mathlarger{\int}_{0}^{ln\,2}\mathlarger{\int}_{0}^{ln\,5}e^{2x-y}\,dx\,dy$

    Si se considera a $y$ como una constante, se obtiene $\mathlarger{\int}_{0}^{ln\,5}e^{2x-y}\,dx$. Si hacemos que $u=2x-y$, entonces
    $$\frac{du}{dx}=2 \qquad \text{ así, } \qquad \frac{du}{2}=dx$$
    Además los límites de integración quedan expresados como $u(\ln{\,5}) = 2(\ln{\,5})-y=2\ln{\,5}-y$ y $u(0)=2(0)-y=-y$. De este modo,
    \begin{align*}
      \mathlarger{\int}_{0}^{ln\,5}e^{2x-y}\,dx
      &= \mathlarger{\int}_{-y}^{2\ln{5}-y}e^{u}\, \frac{du}{2}  \\
      &= \frac{1}{2} \mathlarger{\int}_{-y}^{2\ln{5}-y}e^{u} \, du \\
      &= \frac{1}{2} \cdot e^u \Big| _{-y}^{2\ln{5}-y} \\
      &= \frac{1}{2} \left( e^{2\ln{5}-y}-e^{-y} \right) \\
      &= \frac{1}{2} \left( \frac{e^{2\ln{5}}}{e^y} - \frac{1}{e^y} \right) \\
      &= \frac{1}{2} \left( \frac{(e^{\ln{5}})^2}{e^y} - \frac{1}{e^y} \right) \\
      &= \frac{1}{2} \left( \frac{(5)^2}{e^y} - \frac{1}{e^y} \right) \\
      &= \frac{1}{2} \left( \frac{25}{e^y} - \frac{1}{e^y} \right) \\
      &= \frac{1}{2} \left( \frac{24}{e^y} \right) \\
      &= \frac{12}{e^y}
    \end{align*}
    Si ahora se integra la función respecto a $y$, se obtiene:
    \begin{align*}
      \mathlarger{\int}_{0}^{ln\,2}\mathlarger{\int}_{0}^{ln\,5}e^{2x-y}\,dx\,dy
      &= \mathlarger{\int}_{0}^{ln\,2}\left[\mathlarger{\int}_{0}^{ln\,5}e^{2x-y}\,dx\right]\,dy \\
      &= \mathlarger{\int}_{0}^{ln\,2}\left(\frac{12}{e^y}\right)\,dy \\
      &= 12\mathlarger{\int}_{0}^{ln\,2}e^{-y}\,dy
    \end{align*}
    Si hacemos $u=-y$, entonces  
    $$ \frac{du}{dy}=-1 \qquad \text{ así, } \qquad -du=dy $$
    Además, los límites de integración quedan expresados como $u(\ln{2})=-(\ln{2})=-\ln{2}$ y $u(0)=0$. De este modo,
    \begin{align*}
      12\mathlarger{\int}_{0}^{ln\,2}e^{-y}\,dy
      &= 12\mathlarger{\int}_{0}^{-\ln{2}}-e^{u}\,du \\
      &= -12\mathlarger{\int}_{0}^{-\ln{2}}e^{u}\,du \\
      &= 12\mathlarger{\int}_{-\ln{2}}^{0}e^{u}\,du 
    \end{align*}
    \begin{align*}
      &= 12 \cdot e^u \Big| _{-\ln{2}}^{0} \\
      &= 12 \left(e^{0}-e^{-\ln{2}}\right) \\
      &= 12\left(1-\frac{1}{e^{\ln{2}}}\right)\\
      &= 12\left(1-\frac{1}{2}\right)\\
      &= 12\left(\frac{1}{2}\right)\\
      &= 6
    \end{align*}
    $\therefore \mathlarger{\int}_{0}^{ln\,2}\mathlarger{\int}_{0}^{ln\,5}e^{2x-y}\,dx\,dy=6$
    \begin{figure}[H]
      \centering
      \includegraphics[scale=0.35]{{./img/t1a_3b.png}}
      \label{fig:3b}
      \caption{$e^{2x-y}$}
    \end{figure}
  \end{enumerate}
  % ------------------------------------------------------------------------------------------------------------------------------------------------------------------------------------------------------------------------------
  \question 
  Un cilindro recto no circular tiene su base $D$ en el plano $xy$ y está acotado superiormente por el paraboloide $z=x^2+y^2$. El volumen del sólido es $$V=\mathlarger{\int}_0^1\mathlarger{\int}_0^{y}(x^2+y^2)\,dx\,dy + \mathlarger{\int}_1^2\mathlarger{\int}_0^{2-y}(x^2+y^2)\,dx\,dy$$ 
  Dibuja la región D y expresa el volumen como una sola integral iterada con el orden de integración invertido. Finalmente evalúa la integral para encontrar el volumen.
  \begin{figure}[H]
    \centering
    \begin{tikzpicture}
      \begin{axis}[
          legend pos=north east,
          axis lines = middle,
          xlabel = $x$,
          ylabel = $y$,
        ]
        
        \addplot[
          name path=f,
          thick,
          domain=-1:3, 
          color=red,
        ]
                {x};
                \addlegendentry{$y=x$}
                
                \addplot[
                  name path=g,
                  thick,
                  domain=-1:3, 
                  color=blue,
                ]
                        {2-x};
                        \addlegendentry{$y=2-x$}

                        \addplot[orange!35, opacity=0.4] fill between[of=f and g, soft clip={domain=0:1}];

                        
                        \draw [dashed, opacity=0.4] (axis cs:{1,0}) -- (axis cs:{1,1});
      \end{axis}
    \end{tikzpicture}
    \label{fig:4regiond}
    \caption{Región $D$}
  \end{figure}
  En la Figura \hyperref[fig:4regiond]{6} se ve que $D$ puede escribirse también como una región de tipo \textbf{I} por simplicidad:
  $$D=\{(x,y)~|~0\leq x\leq 1,~ x\leq y \leq 2-x \}$$
  Por tanto, otra expresión para $V$ es
  \begin{align*}
    V
    &=\mathlarger{\int}_0^1\mathlarger{\int}_{x}^{2-x}(x^2+y^2)\,dy\,dx \\
    &=\mathlarger{\int}_0^1\left[x^2y+\frac{y^3}{3}\right]_{x}^{2-x}\,dx\\
    &=\mathlarger{\int}_0^1\left\{\left[x^2(2-x)+\frac{(2-x)^3}{3}\right]-\left[x^2(x)+\frac{(x)^3}{3}\right]\right\}\,dx\\
    &=\mathlarger{\int}_0^1 \left[
      \left(2x^2-x^3+\frac{-x^3+6x^2-12x+8}{3}\right)
      -\left(x^3+\frac{x^3}{3}\right)
      \right]\,dx\\
    &=\mathlarger{\int}_0^1 
    \left(2x^2-x^3+\frac{-x^3+6x^2-12x+8}{3}
    -x^3-\frac{x^3}{3}\right)
    \right]\,dx\\
      &=\mathlarger{\int}_0^1 
      \left(
      2x^2-2x^3+\frac{-2x^3+6x^2-12x+8}{3}
      \right)
      \right]\,dx\\
        &= \mathlarger{\int}_0^1 
        \left(
        2x^2-2x^3-\frac{2}{3}x^3+2x^2-4x+\frac{8}{3}
        \right)
        \right]\,dx\\
          &= \mathlarger{\int}_0^1 
          \left(
          4x^2-\frac{8}{3}x^3-4x+\frac{8}{3}
          \right)
          \right]\,dx\\
            &=\left[
              \frac{4}{3}x^3-\frac{2}{3}x^4-2x^2+\frac{8}{3}x
              \right]_0^1\\
            &=\left[
              \frac{4}{3}(1)^3-\frac{2}{3}(1)^4-2(1)^2+\frac{8}{3}(1)
              \right]
            -\left[
              \frac{4}{3}(0)^3-\frac{2}{3}(0)^4-2(0)^2+\frac{8}{3}(0)
              \right]\\
            &=\left( \frac{4}{3}-\frac{2}{3}-2+\frac{8}{3}\right)-0
  \end{align*}
  \begin{align*}
    &= \frac{4}{3}
  \end{align*}
  $\therefore $ El volumen del sólido es de $\frac{4}{3}$ unidades cúbicas.
  %-------------------------------------------------------------------------------------------------------------------------------------------------------------------------------------------------------------------------
  \question
  Encuentra el volumen de los solidos con las siguientes características:
  \begin{enumerate}[a)]
  \item Acotado por la superficie $z=x\sqrt{x^2+y}$ y los planos $x=0$, $x=1$, $y=0$, $y=1$ y $z=0$.
    
    El volumen requerido se localiza debajo de la gráfica de la función $z=x\sqrt{x^2+y}$ y arriba de
    $$D=\{(x,y)~|~0\leq x\leq 1,~ 0\leq y\leq 1\}$$

    \begin{figure}[H]
      \centering
      \begin{tikzpicture}
        \begin{axis}[
            axis lines = middle,
            xlabel = $x$,
            ylabel = $y$,
            xmin=-0.5, xmax=1.5,
            ymin=-0.5, ymax=1.5,
            xtick={0,1},
            ytick={0,1},
            grid=major,
            legend pos=north east,
            clip=false
          ]
          
          % Sombra del cuadrado
          \addplot [
            draw=none,
            fill=orange,
            fill opacity=0.3,
          ]
          coordinates {(0,0) (1,0) (1,1) (0,1) (0,0)};
          
          % Dibujar el borde del cuadrado
          \addplot [
            thick,
            color=blue,
          ]
          coordinates {(0,0) (1,0) (1,1) (0,1) (0,0)};
          
          % Etiquetas de las líneas
          \node at (axis cs: 0.2,-0.2) [below] {$x=0$};
          \node at (axis cs: 1.2,-0.2) [below] {$x=1$};
          \node at (axis cs: -0.2,0.2) [left] {$y=0$};
          \node at (axis cs: -0.2,1.2) [left] {$y=1$};

        \end{axis}
      \end{tikzpicture}
      \label{fig:t1_5a}
      \caption{Región $D$}
    \end{figure}
    
    Por consiguiente,
    \begin{align*}
      V
      &= \mathlarger{\iint}_D \left(x\sqrt{x^2+y}\right)\,dA \\
      &= \mathlarger{\int}_0^1\mathlarger{\int}_0^1\left(x\sqrt{x^2+y}\right)\,dy\,dx\\
      &= \mathlarger{\int}_0^1\left[\mathlarger{\int}_0^1\left(x\sqrt{x^2+y}\right)\,dy\right]\,dx
    \end{align*}
    Si hacemos $u=x^2+y~\rightarrow ~ \frac{du}{dy}=1$, así $du=dy$. Además, $u(1)=x^2+1$ y $u(0)=x^2$. Entonces,
    \begin{align*}
      \mathlarger{\int}_0^1\left[\mathlarger{\int}_0^1\left(x\sqrt{x^2+y}\right)\,dy\right]\,dx
      &= \mathlarger{\int}_0^1\left[\mathlarger{\int}_{x^2}^{x^2+1}\left(x\sqrt{u}\right)\,du\right]\,dx\\
      &= \mathlarger{\int}_0^1\left[x\mathlarger{\int}_{x^2}^{x^2+1}u^{\frac{1}{2}}\,du\right]\,dx\\
      &= \mathlarger{\int}_0^1\left\{\frac{2}{3}x\left[(x^2+1)^{\frac{3}{2}}-(x^2)^{\frac{3}{2}}\right]\right\}\,dx\\
      &=\frac{2}{3}\mathlarger{\int}_0^1\left[x(x^2+1)^{\frac{3}{2}}-x^3\right]\,dx
    \end{align*}
    
    %-------------------------------------------------------------------------------------------------------------
  \item Encerrado por los cilindros $z=x^2$, $y=x^2$ y los planos $z=0$ y $y=4$.
    Puesto que el segundo cilíndro parabólico corta al plano $xy$ (cuya ecuación es $z=0$) en la parábola $y=x^2$ y el segundo plano en la recta $y=4$. Se ve que el volumen $V$ está arriba de la región $D$ en el plano $xy$ acotado por la recta $y=4$ y la parábola $y=x^2$. (Véase la figura \hyperref[fig:5regionb]{8})
    \begin{figure}[H]
      \centering
      \begin{tikzpicture}
        \begin{axis}[
            legend pos=north east,
            axis lines = middle,
            xlabel = $x$,
            ylabel = $y$,
            domain=-3:3,
          ]
          
          \addplot[
            name path=f,
            thick,
            color=red,
          ]
                  {x^2};
                  \addlegendentry{$y=x^2$}
                  
                  \addplot[
                    name path=g,
                    thick, 
                    color=blue,
                  ]
                          {4};
                          \addlegendentry{$y=4$}

                          \addplot[orange!35, opacity=0.4] fill between[of=f and g, soft clip={domain=-2:2}];

                          
                          \draw [dashed, opacity=0.4] (axis cs:{2,0}) -- (axis cs:{2,4});
                          \draw [dashed, opacity=0.4] (axis cs:{-2,0}) -- (axis cs:{-2,4});
        \end{axis}
      \end{tikzpicture}
      \label{fig:5regionb}
      \caption{Región $D$}
    \end{figure}
    Así que el volumen requerido se localiza debajo de la gráfica de la función $z=x^2$ y arriba de
    $$D=\{(x,y)~|~-\sqrt{y} \leq x \leq \sqrt{y},~ 0\leq y\leq 4\}$$
    Por consiguiente,
    \begin{align*}
      V
      &= \mathlarger{\int}_0^4\mathlarger{\int}_{-\sqrt{y}}^{\sqrt{y}}x^2\,dx\,dy \\
      &= \mathlarger{\int}_0^4\left[\mathlarger{\int}_{-\sqrt{y}}^{\sqrt{y}}x^2\,dx\right]\,dy \\
      &= \mathlarger{\int}_0^4\left[\frac{x^{2+1}}{2+1}\right]_{-\sqrt{y}}^{\sqrt{y}}\,dy \\
      &= \mathlarger{\int}_0^4\left[\frac{x^3}{3}\right]_{-\sqrt{y}}^{\sqrt{y}}\,dy \\
      &= \mathlarger{\int}_0^4\left\{\left[\frac{(\sqrt{y})^3}{3}\right]-\left[\frac{(-\sqrt{y})^3}{3}\right]\right\}\,dy \\
      &= \mathlarger{\int}_0^4\left(\frac{y^{\frac{3}{2}}}{3}+\frac{y^{\frac{3}{2}}}{3}\right)\,dy\\
      &= \frac{2}{3}\mathlarger{\int}_0^4y^{\frac{3}{2}}\,dy
    \end{align*}
    \begin{align*}
      &= \frac{2}{3}\cdot \frac{y^{\frac{3}{2}+1}}{\frac{3}{2}+1} \Big|_0^4 \\
      &= \frac{2}{3}\cdot \frac{y^{\frac{3}{2}+1}}{\frac{5}{2}} \Big|_0^4 \\
      &=\frac{2}{3}\cdot \frac{2}{5}\cdot y^{\frac{5}{2}} \Big|_0^4 \\
      &=\frac{4}{15}\cdot \left[(4)^{\frac{5}{2}}-(0)^{\frac{5}{2}}\right] \\
      &=\frac{4}{15}\cdot \left(\sqrt{4^5}- 0) \\
      &=\frac{4}{15}\cdot \sqrt{1024} \\
      &=\frac{4}{15}\cdot 32 \\
      &=\frac{128}{15}
    \end{align*}
    \begin{figure}[H]
      \centering
      \includegraphics[width=0.5\textwidth]{{./img/t1a_5b.png}}
      \label{fig:5b}
      \caption{Sólido $5b)$}
    \end{figure}
    $\therefore$ El volúmen del sólido es de $\frac{128}{15}$ unidades cúbicas.
    
  \end{enumerate}

  \question
  Evalúa la integral invirtiendo el orden de integración.
  \begin{enumerate}[a)]
  \item $\mathlarger{\int}_0^1\mathlarger{\int}_{3y}^3 e^{x^2}\,dx\,dy$

    La región $D$ acotada por la recta $x=3y$ y $x=3$ con $y=0$ y $y=1$ se muestra en la figura \hyperref[fig:6regiona]{10}
    \begin{figure}[H]
      \centering
      \begin{tikzpicture}
        \begin{axis}[
            legend pos=north east,
            axis lines = middle,
            xlabel = $x$,
            ylabel = $y$,
            domain=0:3,
            ymin=0, ymax=2,
          ]
          
          % Plot the line y = 1/3 * x
          \addplot[
            name path=h,
            thick,
            color=red,
          ]
                  {1/3*x};
                  \node[color=red] at (axis cs: 1,0.7) {$y=\frac{1}{3}x$};
                  
                  % Plot the vertical line x = 3
                  \addplot[
                    name path=p,
                    thick, 
                    color=blue,
                  ]
                  coordinates {(3,-1) (3,2)};
                  \node[color=blue] at (axis cs: 2.7,1.5) {$x=3$};

                  % Shade the area between the curves
                  \addplot[orange!35, opacity=0.4] fill between[
                    of=h and p
                  ];

                  \draw [dashed, opacity=0.4] (axis cs:{0,1}) -- (axis cs:{3,1});

        \end{axis}
      \end{tikzpicture}
      \label{fig:6regiona}
      \caption{Región $D$}
    \end{figure}
    Si se hubiera expresado a $D$ como una región \textbf{tipo I}, entonces se abría obtenido:
    $$D=\left\{(x,y)~|~0 \leq x\leq 3,~ 0 \leq y\leq \frac{1}{3}x\right\}$$
    Con lo cual,
    \begin{align*}
      \iint\limits_{\underset{\textstyle D}{}} e^{x^2}\, dA
      &= \mathlarger{\int}_0^3\mathlarger{\int}_0^{\frac{1}{3}x} e^{x^2}\, dy\,dx \\
      &= \mathlarger{\int}_0^3\left[\mathlarger{\int}_0^{\frac{1}{3}x} e^{x^2}\, dy\right]\,dx \\
      &= \mathlarger{\int}_0^3\left[e^{x^2}y\right]_0^{\frac{1}{3}x} \,dx \\
      &= \mathlarger{\int}_0^3\frac{1}{3}xe^{x^2} \,dx 
    \end{align*}
    Si hacemos $u=x^2 ~ \rightarrow ~ \frac{du}{dx}=2x$, así $x\,dx = \frac{du}{2}$. Asímismo $u(3)=9$ y $u(0)=0$. Entonces,
    \begin{align*}
      \iint\limits_{\underset{\textstyle D}{}} e^{x^2}\, dA
      &= \mathlarger{\int}_0^3\frac{1}{3}xe^{x^2} \,dx  \\
      &= \frac{1}{6} \int_0^9 e^u du \\
      &= \frac{1}{6} \cdot e^u \Big|_0^9\\
      &= \frac{1}{6} \cdot e^9-1
    \end{align*}
    $\therefore$ Invirtiendo el orden de integración tenemos que $\mathlarger{\int}_0^3\mathlarger{\int}_0^{\frac{1}{3}x} e^{x^2}\, dy\,dx = \frac{e^9-1}{6}$
    
  \item $\mathlarger{\int}_0^1\mathlarger{\int}_{\text{arcsen}\,y}^{\pi/2}\text{cos}(x)\sqrt{1+\text{cos}^2x}\,dx\,dy$

    La región $D$ acotada por la función $x=\arcsen{(y)}$ y $x=\frac{\pi}{2}$ con $y=0$ y $y=1$ se muestra en la figura \hyperref[fig:6regionb]{11}
    \begin{figure}[H]
      \centering
      \begin{tikzpicture}
        \begin{axis}[
            axis lines = middle,
            xlabel = $x$,
            ylabel = $y$,
            domain=0:pi,
            samples=100,
            ymin=0, ymax=1.5,
            xmin=0, xmax=3.5,
            xtick={0, pi/2, pi},
            xticklabels={$0$, $\frac{\pi}{2}$, $\pi$},
            legend pos=north east
          ]
          
          % Plot the sine function y = sin(x)
          \addplot[
            name path=sine,
            thick,
            color=blue,
          ]
                  {sin(deg(x))};
                  \addlegendentry{$x=\arcsen{y}$}
                  
                  % Plot the vertical line x = pi/2
                  \addplot[
                    name path=line,
                    thick,
                    color=red,
                  ]
                  coordinates {(pi/2,-1.5) (pi/2,1.5)};
                  \addlegendentry{$x = \frac{\pi}{2}$}

                  % Plot the x-axis (y=0) as a path
                  \addplot[
                    name path=axis,
                    draw=none,
                    domain=0:pi/2,
                  ] 
                          {0};
                          
                          % Shade the area above y=0
                          \addplot [
                            thick,
                            color=orange,
                            fill=orange,
                            fill opacity=0.3,
                          ]
                          fill between[
                            of=sine and axis,
                            soft clip={domain=0:pi/2},
                          ];
                          \draw [dashed, opacity=0.4] (axis cs:{0,1}) -- (axis cs:{pi/2,1});

        \end{axis}
      \end{tikzpicture}
      \label{fig:6regionb}
      \caption{Región $D$}
    \end{figure}

    Si se hubiera expresado a $D$ como una región \textbf{tipo I}, entonces se abría obtenido:
    $$D=\left\{(x,y)~|~0 \leq x\leq \frac{\pi}{2},~ 0 \leq y\leq \sin{x} \right\}$$

    Con lo cual,
    \begin{align*}
      \iint\limits_{\underset{\textstyle D}{}} \text{cos}(x)\sqrt{1+\text{cos}^2x}\, dA
      &= \mathlarger{\int}_0^{\frac{\pi}{2}}\mathlarger{\int}_0^{\sin{x}} \text{cos}(x)\sqrt{1+\text{cos}^2x}\, dy\,dx \\
      &= \mathlarger{\int}_0^{\frac{\pi}{2}}\left[ \text{cos}(x)\sqrt{1+\text{cos}^2x}\cdot y\right]_0^{\sin{x}}\,dx \\
      &= \mathlarger{\int}_0^{\frac{\pi}{2}}\left( \text{sen}(x)\text{cos}(x)\sqrt{1+\text{cos}^2x}\right)\,dx 
    \end{align*}

    Si hacemos $u=\cos{x} ~\rightarrow~ \frac{du}{dx}=-\sin{x}$, así $-du=\sin{x\,dx}$. Además, $u(0)=1$ y $u(\frac{\pi}{2})=1$. De modo que,
    \begin{align*}
      \mathlarger{\int}_0^{\frac{\pi}{2}}\left( \text{sen}(x)\text{cos}(x)\sqrt{1+\text{cos}^2x}\right)\,dx
      &= 
      \mathlarger{\int}_1^0\left( -u\sqrt{1+u^2}\right)\,du \\
      &= \mathlarger{\int}_0^1\left( u\sqrt{1+u^2}\right)\,du \\
    \end{align*}
    Si hacemos $a=\sqrt{1+u^2}~\rightarrow ~ \frac{da}{du}=\frac{1}{2}(1+u^2)^{-\frac{1}{2}}\cdot 2u= \frac{u}{\sqrt{1+u^2}}$, así $u\,du = \sqrt{1+u^2}\, da=a\,da$. Además, $a(1)=\sqrt{2}$ y $a(0)=1$. Entonces, 
    \begin{align*}
      \mathlarger{\int}_0^1\left( u\sqrt{1+u^2}\right)\,du
      &= \mathlarger{\int}_1^{\sqrt{2}} (a\cdot a)\,da\\
      &= \mathlarger{\int}_1^{\sqrt{2}} a^2\,da\\
      &= \frac{a^3}{3}\Big|_1^{\sqrt{2}}\\
      &= \frac{2\sqrt{2}}{3}- \frac{1}{3}\\
      &= \frac{1}{3}\left(2\sqrt{2}-1\right)
    \end{align*}

    $\therefore$  Invirtiendo el orden de integración tenemos que $\mathlarger{\int}_0^{\frac{\pi}{2}}\mathlarger{\int}_0^{\sin{x}} \text{cos}(x)\sqrt{1+\text{cos}^2x}\, dy\,dx = \frac{1}{3}\left(2\sqrt{2}-1\right)$
  \end{enumerate}

  \question
  Encuentra el volumen del sólido $S$ como la diferencia entre dos volúmenes. $S$ es el sólido encerrado por los cilindros parabólicos $y=1-x^2$, $y=x^2-1$ y las planos $x+y+z=2$, $2x+2y-z+10=0$.

  \begin{itemize}
  \item Dado que el plano $x+y+z=2$ intersecta al plano $xy$ en $z=0$, se tiene que $x+y=2~\rightarrow ~y=2-x$ es la línea de intersección.
  \item Dado que el plano $2x+2y-z+10=0$ intersecta al plano $xy$ en $z=0$, se tiene que $2x+2y=-10~\rightarrow ~y=-5-x$ es la línea de intersección.
  \item Dado que los planos $x+y+z=2$ y $2x+2y-z+10=0$ se intersectan cuando los igualamos en $z$, se tiene que $2-x-y=2x+2y+10~\rightarrow ~ 3x+3y=-8 ~ \rightarrow ~ y = -x-\frac{8}{3}$ es la línea de intersección.
    \begin{itemize}
    \item Nótese que el plano $2x+2y-z+10=0$ se encuentra por encima del plano $x+y+z=2$ en el dominio
      $$D=\left\{(x,y)~|~-1\leq x\leq 1, ~ x^2-1\leq y \leq 1-x^2\right\}$$
    \end{itemize}
  \end{itemize}

  Así, se obtiene la región $D$ de integración del sólido $S$.
  
  \begin{figure}[H]
    \centering
    \begin{tikzpicture}
      \begin{axis}[
          axis lines = middle,
          xlabel = $x$,
          ylabel = $y$,
          domain=-3:3,
          samples=100,
          ymin=-6, ymax=3,
          xmin=-2.5, xmax=2.5,
          legend pos=south east,
          clip=false
        ]
        
        % Plot y = 1 - x^2
        \addplot[
          name path=curve1,
          thick,
          color=blue,
        ]
                {1 - x^2};
                \node[color=blue] at (axis cs: 2.7,-2) {$y = 1 - x^2$};
                
                % Plot y = x^2 - 1
                \addplot[
                  name path=curve2,
                  thick,
                  color=red,
                ]
                        {x^2 - 1};
                        \node[color=red] at (axis cs: 2.7,2) {$y = x^2 - 1$};

                        % Shade the area between y = 1 - x^2 and y = x^2 - 1
                        \addplot [
                          thick,
                          color=orange,
                          fill=orange,
                          fill opacity=0.3,
                        ]
                        fill between[
                          of=curve1 and curve2,
                          soft clip={domain=-1:1},
                        ];

                        % Plot y = 2 - x
                        \addplot[
                          thick,
                          color=green,
                        ]
                                {2 - x};
                                \node[color=green] at (axis cs: -0.5,4) {$y = 2 - x$};
                                
                                % Plot y = -5 - x
                                \addplot[
                                  thick,
                                  color=purple,
                                ]
                                        {-5 - x};
                                        \node[color=purple] at (axis cs: 1,-7.5) {$y = -5 - x$};

                                        % Plot y = -x - 8/3
                                        \addplot[
                                          thick,
                                          color=brown,
                                        ]
                                                {-x - 8/3};
                                                \node[color=brown] at (axis cs: 3.5,-4.4) {$y = -x - \frac{8}{3}$};

      \end{axis}
    \end{tikzpicture}
    \label{fig:t1a_7}
    \caption{Región $D$}
  \end{figure}

  Luego, el plano $2x+2y-z+10=0$ se puede escribir como $z=-2x-2y-10$ y el plano $x+y+z=2$ puede escribirse como $z=-x-y+2$ también. De este modo, el volumen $V_s$ del sólido $S$ requerido se localiza debajo de la función  $z=-2x-2y-10$ y arriba de la función $x+y+z=2$ en la región $D$.
  
  \begin{figure}[H]
    \centering
    \includegraphics[scale=0.4]{{./img/t1a_7.png}}
    \label{fig:5b}
    \caption{Sólido $5b)$}
  \end{figure}

  Ahora, expresamos el volumen $V_s$ del sólido $S$ como la diferencia entre dos volúmenes
  \begin{align*}
    V_s
    &= \mathlarget{\int}_{-1}^1\mathlarget{\int}_{x^2-1}^{1-x^2} \left(2x+2y+10\right)\,dy\,dx
    - \mathlarget{\int}_{-1}^1\mathlarget{\int}_{x^2-1}^{1-x^2} \left(-x-y+2\right)\,dy\,dx\\
    &= \mathlarget{\int}_{-1}^1\mathlarget{\int}_{x^2-1}^{1-x^2} \left[(2x+2y+10)-(-x-y+2)\right]\,dy\,dx\\
    &= \mathlarget{\int}_{-1}^1\mathlarget{\int}_{x^2-1}^{1-x^2} (2x+2y+10+x+y-2)\,dy\,dx\\
    &= \mathlarget{\int}_{-1}^1\left[2xy+y^2+10y+xy+\frac{y^2}{2}-2y\right]_{x^2-1}^{1-x^2}\,dx\\
    &= \mathlarget{\int}_{-1}^1\left[3xy+8y+\frac{3y^2}{2}\right]_{x^2-1}^{1-x^2}\,dx\\
    &= \mathlarget{\int}_{-1}^1\left\{\left[3x(1-x^2)+8(1-x^2)+\frac{3(1-x^2)^2}{2}\right]
    - \left[3x(x^2-1)+8(x^2-1)+\frac{3(x^2-1)^2}{2}\right]\right\}\,dx\\
    &= \mathlarget{\int}_{-1}^1 [ \left(3x-3x^3+8-8x^2+\frac{3}{2}-\frac{3}{2}x^2+\frac{3}{2}x^4\right)\\ &\qquad- \left(3x^3-3x+8x^2-8+\frac{3}{2}x^4-\frac{3}{2}x^2+\frac{3}{2}\right) ] \,dx\\
    &= \mathlarget{\int}_{-1}^1 \left(-6x^3+6x-16x^2+16 \right)\,dx\\
    &= \left[-\frac{3}{2}x^4+3x^2-\frac{16}{3}x^3+16x\right]_{-1}^1\\
    &= \left[-\frac{3}{2}+3-\frac{16}{3}+16\right] - \left[-\frac{3}{2}+3+\frac{16}{3}-16\right]\\
    &= -\frac{32}{3}+32\\
    &=\frac{64}{3}
  \end{align*}

  $\therefore$ El volumen del sólido $S$ es $\frac{64}{3}\approx 21.3333$ unidades cúbicas.
  
\end{questions}
\vskip30pt
\RaggedRight

\newpage


\newgeometry {
  hmargin = {1.5cm, 1.5cm},
  vmargin = {5cm, 1cm},
  nohead,			% Elimina el encabezado
  nomarginpar,	% Elimina las notas
  includeall,
}% \savegeometry{geometria_1}

\pagestyle{foot}    % El estilo de ésta página sólo constará de pié de página
\runningfooter{}{}{Página \thepage\ de \numpages}

\end{document}
