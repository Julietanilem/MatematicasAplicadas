\documentclass[12pt]{article}

\usepackage[spanish]{babel}

\usepackage{amsmath}
\usepackage{amssymb}

\usepackage{hyperref}
\usepackage{graphicx}
\usepackage{listings}
\usepackage{color}
\usepackage{multicol}
\usepackage{enumitem}
\usepackage{here}
\usepackage{dsfont}
\usepackage{tipa}
\usepackage{float}
\spanishdecimal{.}

\title{Matemáticas para las Ciencias Aplicadas II}
\title{
	\textbf{Tarea 01} \\
	\vspace{1ex}
	\large Matemáticas para las Ciencias Aplicadas II \\
	Facultad de Ciencias, UNAM}
\date{\today}
\author{Flores Morán Julieta Melina \\ Zarco Romero José Antonio}

\begin{document}
\maketitle

% 1 -------------------------------------------------------------------------------------------------------------
\section{}

¿Cuáles son las proyecciones del \textbf{punto} $A(2,3,5)$ en los planos \textbf{xy} \textbf{yz} y \textbf{xz}. Para calcular, trace una caja rectangular con vértices en el origen y el punto A como vértices opuestos y con sus caras paralelas a los planos coordenados. Etiquete todos los vertices de la caja. Asimismo calcule la longitud de la diagonal de la caja.

% 2 -------------------------------------------------------------------------------------------------------------
\section{}

Determine una ecuación de la esfera que pasa por el origen y cuyo centro es el punto $A(1,2,3)$. Describa su intersección con cada uno de los planos coordenados.

El radio será la distancia entre los puntos $O(0,0,0)$ (origen) y $A(1,2,3)$ (centro de la esfera), i.e.

\begin{align*}
  r = |OA|
  &= \sqrt{(1-0)^2 + (2-0)^2 + (3-0)^2} \\
  &= \sqrt{1^2 + 2^2 + 3^2} \\
  &= \sqrt{1 + 4 + 9} \\
  &= \sqrt{14}
\end{align*}

Por lo tanto, la ecuación de la esfera es: $$ (x-1)^2 + (y-2)^2 + (z-3)^2 = 14 $$

\begin{enumerate}[format=\textbf]
  
\item Intersección con el plano $xy$.
  
  Sustituimos $z = 0$ en la ecuación de la esfera.
  
  \begin{align*}
    (x-1)^2 + (y-2)^2 + (0-3)^2
    &= 14 \\
    (x-1)^2 + (y-2)^2 + (-3)^2
    &= 14 \\
    (x-1)^2 + (y-2)^2 + 9
    &= 14 \\
    (x-1)^2 + (y-2)^2
    &= 14 - 9 \\
    (x-1)^2 + (y-2)^2
    &= 5 \\
  \end{align*}
  
  Como $z = 0$, los puntos que se encuentran en el plano $xy$ se hallan sobre la circunferencia $(x-1)^2 + (y-2)^2 = 5$.
  
\item Intersección con el plano $xz$.
  
  Sustituimos $y = 0$ en la ecuación de la esfera.
  
  \begin{align*}
    (x-1)^2 + (0-2)^2 + (z-3)^2
    &= 14 \\
    (x-1)^2 + (-2)^2 + (z-3)^2
    &= 14 \\
    (x-1)^2 + 4 + (z-3)^2
    &= 14 \\
    (x-1)^2 + (z-3)^2
    &= 14 - 4 \\
    (x-1)^2 + (z-3)^2
    &= 10 \\
  \end{align*}
  
  Como $y = 0$, los puntos que se encuentran en el plano $xz$ se hallan sobre la circunferencia $(x-1)^2 + (z-3)^2 = 10$.
  
\item Intersección con el plano $yz$.
  
  Sustituimos $x = 0$ en la ecuación de la esfera.
  
  \begin{align*}
    (0-1)^2 + (y-2)^2 + (z-3)^2
    &= 14 \\
    (-1)^2 + (y-2)^2 + (z-3)^2
    &= 14 \\
    1 + (y-2)^2 + (z-3)^2
    &= 14 \\
    (y-2)^2 + (z-3)^2
    &= 14 - 1 \\
    (y-2)^2 + (z-3)^2
    &= 13 \\
  \end{align*}
  
  Como $x = 0$, los puntos que se encuentran en el plano $yz$ se hallan sobre la circunferencia $(y-2)^2 + (z-3)^2 = 13$.
  
\end{enumerate}

% 3 -------------------------------------------------------------------------------------------------------------
\section{}

La siguiente ecuación corresponde a una esfera. Determine las coordenadas del centro y el radio.
\[ x^2 + y^2 + z^2 + 8x - 6y + 2z + 17 = 0 \]

Se puede reescribir la ecuación dada en la forma de la ecuación de una esfera si se completan los cuadrados:

\begin{equation*}
  \begin{split}
    x^2 + y^2 + z^2 + 8x - 6y + 2z + 17 &= 0 \\
    x^2 + 8x + y^2 - 6y + z^2 + 2z + 17 &= 0 \\
    (x^2 + 8x) + (y^2 - 6y) + (z^2 + 2z) &= - 17 \\
    (x^2 + 8x + 16) + (y^2 - 6y + 9) + (z^2 + 2z + 1) &= - 17 + 16 + 9 + 1\\
    (x + 4)^2 + (y - 3)^2 + (z + 1)^2 &= 9 \\
  \end{split}
\end{equation*}

$\therefore $ Se ve que es la ecuación de una esfera con centro $(-4, 3, -1)$ y radio $\sqrt{9} = 3$.

% 4 -------------------------------------------------------------------------------------------------------------
\section{}

Escriba desigualdades para describir las siguientes regiones.

\begin{itemize}
    
\item La región entre el plano \textbf{xz} y el plano vertical \textbf{y=4}.

  Dado que la ecuación del plano $xz$ es $y=0$  y la del plano vertical es $y=4$. La región que delimitan incluye a todos los puntos $(x,y,x)$ donde
  $$0 \leq y \leq 4$$
    
\item La región que consta de todos los puntos entre (pero no sobre) las esferas de radio \textit{r} y \textit{R} centradas en el origen, donde $r < R$.

  Para representar a los puntos $(x,y,z)$ cuya distancia desde el origen es por lo menos $r$, y a lo más, $R$, podemos delimitar a la región como
  $$r < \sqrt{x^2 + y^2 + z^2} < R$$
  o bien,
  $$r^2 < x^2 + y^2 + z^2 < R^2$$
    
\end{itemize}
  
% 5 -------------------------------------------------------------------------------------------------------------
\section{}

Sean $\vec{a} , \vec{b} , \vec{c}$ vectores en $\mathbb{R}^n$ y sean $c,d$ escalares. Escriba las 8 propiedades de los vectores y proporcione una breve explicación de cada una de ellas.

\begin{enumerate}

\item $$\vec{a}+\vec{b} = \vec{b}+\vec{a}$$

\item $$\vec{a}+\vec{0} = \vec{a}$$

\item $$c(\vec{a}+\vec{b}) = c\vec{a}+c\vec{b}$$

\item $$(cd)\vec{a} = c(d\vec{a})$$

\item $$\vec{a}+(\vec{b}+\vec{c}) = (\vec{a}+\vec{b})+\vec{c}$$

\item $$\vec{a}+(\vec{-a}) = \vec{0}$$

\item $$(c+d)\vec{a} = c\vec{a}+d\vec{a}$$

\item $$1\vec{a} = \vec{a}$$

\end{enumerate}

% 6 -------------------------------------------------------------------------------------------------------------
\section{}
Obtenga un vector $\vec{a}$, como el segmento de recta dirigida de $\vec{AB}$ , donde \textit{A} y \textit{B} son los puntos:

\begin{itemize}
  
\item $A(-5,-1) , B(-3,3)$

  El vector correspondiente a $\vec{AB}$ es

  \begin{align*}
    \vec{a}
    &=
    \langle
    [(-3)-(-5)],
    [3-(-1)]
    \rangle \\
    &=
    \langle
    (-3+5),
    (3+1)
    \rangle \\
    &=
    \langle
    2,
    4
    \rangle \\
  \end{align*}

  $\therefore \vec{a} = \langle 2, 4 \rangle$
  
\item $A(0,6,1), B(3,4,4)$

  El vector correspondiente a $\vec{AB}$ es

  \begin{align*}
    \vec{a}
    &=
    \langle
    (3-0),
    (4-6),
    (4-1)
    \rangle \\
    &=
    \langle
    3,
    -2,
    3
    \rangle \\
  \end{align*}

  $\therefore \vec{a} = \langle 3, -2, 3 \rangle$
  
\end{itemize}

Haga un esbozo (en cada caso) del vector $\vec{AB}$ y la representación \textbf{equivalente} comenzando en el origen.

% 7 -------------------------------------------------------------------------------------------------------------
\section{}

Determine (donde $\vec{a}=8\hat{i} + \hat{j} - 4\hat{k}$; $\vec{b}= 5\hat{i} - 2\hat{j} +\hat{k}$)

\begin{enumerate}[label=\alph*), format=\textbf]
  
\item $\vec{a} + \vec{b}$
  \begin{equation*}
    \begin{split}
      \vec{a} + \vec{b}
      &= (8\hat{i} + \hat{j} - 4\hat{k}) + (5\hat{i} - 2\hat{j} +\hat{k}) \\
      &= 13\hat{i} - 1\hat{j} - 3\hat{k} \\
    \end{split}
  \end{equation*}
  
\item $4 \vec{a} + 2\vec{b}$
  \begin{equation*}
    \begin{split}
      4 \vec{a} + 2\vec{b}
      &= 4 (8\hat{i} + \hat{j} - 4\hat{k}) + 2 (5\hat{i} - 2\hat{j} +\hat{k}) \\
      &= (32\hat{i} + 4\hat{j} - 16\hat{k}) + (10\hat{i} - 4\hat{j} + 2\hat{k}) \\
      &= 42\hat{i} + 0\hat{j} - 14\hat{k} \\
      &= 42\hat{i} - 14\hat{k}
    \end{split}
  \end{equation*}

\item $|\vec{a} - \vec{b}|$

\item $|\vec{a}|$
  \begin{equation*}
    \begin{split}
      |\vec{a}|
      &= | 8\hat{i} + \hat{j} - 4\hat{k} | \\
      &= \sqrt{(8)^2 + (1)^2 + (-4)^2} \\
      &= \sqrt{64 + 1 + 16} \\
      &= \sqrt{81}
    \end{split}
  \end{equation*}

\end{enumerate}

% 8 -------------------------------------------------------------------------------------------------------------
\section{}

Sea $\vec{a}$ un vector tal que se ubica en el primer cuadrante, hace un ángulo de $\frac{\pi}{6}$ con el eje $x$ positivo  y $|\vec{a}|=2$. Determine $\vec{a}$ en términos de sus componentes.

% 9 -------------------------------------------------------------------------------------------------------------
\section{}

Un vendedor ambulante vende $a$ hamburguesas, $b$ hot dogs y $c$ refrescos en un día dado. Cobra 4 pesos por hamburguesas, 2.5 pesos por hot dog y 1 peso pro refresco. Sea $\vec{a}=(a,b,c)$ y $\vec{P}=(4,2.5,1)$. ¿Qué representa el producto punto $\vec{a} \cdot \vec{P}$.

% 10 ------------------------------------------------------------------------------------------------------------
\section{}

Encuentre las proyecciones escalar y vectorial de $\vec{b}$ sobre $\vec{a}$

\begin{itemize}
  
\item $\vec{a}= (5,12), \vec{b}=(4,6)$
  
\item $\vec{a}= (1,4), \vec{b}=(2,3)$
  
\end{itemize}

% 11 ------------------------------------------------------------------------------------------------------------
\section{}

Dado los vectores $\vec{a} = \hat{i} + 2\hat{j}-2\hat{k}, \vec{b} = 4\hat{i} -3\hat{k}$ \\
Calcule el ángulo entre los vectores:

\begin{itemize}

\item En grados

\item En radianes

\end{itemize}

% 12 ------------------------------------------------------------------------------------------------------------
\section{}

Dados los vectores \\

\[\vec{a} = \hat{j}+ 7\hat{k}\]

\[\vec{b} = 2\hat{i} - \hat{j}+ 4\hat{k}\]

obtenga:

\begin{itemize}

\item $\vec{c}=\vec{a} \times \vec{b}$

\item Compruebe que $\vec{c}$ es ortogonal a $\vec{a}$ y $\vec{b}$ simultáneamente.

\end{itemize}

% 13 ------------------------------------------------------------------------------------------------------------
\section{}

Proporcione:

\begin{enumerate}

\item La ecuación vectorial

\item Las ecuaciones paramétricas

\item Las ecuaciones simétricas para las siguientes rectas:

  \begin{itemize}

  \item La recta que pasa por $P(6,5,2)$ y que esperalela al vector $\vec{u}=(1,3,\frac{-2}{3})$

  \item La recta que pasa por $A(0,0,0)$ y $B(4,3,-1)$

  \end{itemize}

\end{enumerate}

% 14 ------------------------------------------------------------------------------------------------------------
\section{}

Utilice el \textbf{triple punto escalar} (producto mixto) para determinar si los puntos $A(1,3,2) , B(3,-1,6), C(5,2,0), D(3,6,-4)$ son coplanares.

% 15 ------------------------------------------------------------------------------------------------------------
\section{}

Proporcione la ecuación del plano que pasa por $A(5,3,5)$y cuyo vector normal es $\vec{n}=2\hat{i} + \hat{j} - \hat{k}$. Adjunte una imágen de geogebra en la situación.

% 16 ------------------------------------------------------------------------------------------------------------
\section{}

Proporcione la ecuación del plano que contiene a los puntos $A(0,0,0) , B(2,-4,6), C(5,1,3)$. Adjunte una imágen de geogebra en la situación.

% 17 ------------------------------------------------------------------------------------------------------------
\section{}

Proporcione las coordenadas del punto $A(a_x,a_y,a_x)$ del punto donde se intersecan:\\

\textbf{el plano} \\

\[ x + 2y -z+1 =0 \]

y \textbf{la recta dada por las ecuaciones paramétricas}

\begin{align*}
  x &= 1 + 2t,\\
  y &=4t,\\
  z &= 2 - 3t
\end{align*}

Adjunte una imágen de geogebrea de la situación.

\end{document}
