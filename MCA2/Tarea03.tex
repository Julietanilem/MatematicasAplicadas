\documentclass[12pt]{article}

\usepackage[spanish]{babel}

\usepackage{amsmath}
\usepackage{amssymb}

\usepackage{hyperref}
\usepackage{graphicx}
\usepackage{listings}
\usepackage{color}
\usepackage{multicol}
\usepackage{enumitem}
\usepackage{here}
\usepackage{dsfont}
\usepackage{tipa}
\usepackage{float}
\usepackage{dsfont} 
\spanishdecimal{.}

\title{Matemáticas para las Ciencias Aplicadas II}
\title{
        \textbf{Tarea 03} \\
        \vspace{1ex}
        \large Matemáticas para las Ciencias Aplicadas II \\
        Facultad de Ciencias, UNAM}
\date{\today}
\author{Flores Morán Julieta Melina \\ Zarco Romero José Antonio}

\begin{document}
\maketitle

% 1 -------------------------------------------------------------------------------------------------------------
\section{}

Sea la función escalar de variable vectorial $$f(x,y,z)= \sqrt{x} +\sqrt{y} +\sqrt{z} + \ln{(4-x^2-y^2-z^2)}$$.

\begin{itemize}[format=\textbf]

\item Evalúe $f(1,1,1)$.

\begin{align*}
f(1,1,1)
&= \sqrt{1} +\sqrt{1} +\sqrt{1} + \ln{(4-1^2-1^2-1^2)} \\
&= 1+1+1+ \ln{(4-1-1-1)} \\
&= 1+1+1+ \ln{1} \\
&= 1+1+1+0 \\
&= 3
\end{align*}

\item Determine y describa el dominio de $f(x, y, z)$.

\end{itemize}

% 2 -------------------------------------------------------------------------------------------------------------
\section{}

 Se muestran las curvas de nivel \textbf{isotermas} para la temperatura del agua $[C]$ en un lago en el año 1998 como una función de la profundidad y el tiempo en años. Estime la temperatura en el lago el 9 de junio (día 160) a una profundidad de 10 \textit{metros} y el día 29 de junio (día 180) a una profundidad de 5 \textit{metros}.

% 3 -------------------------------------------------------------------------------------------------------------
\section{}

Determine los siguientes \textbf{límites}, si existen, o demuestren que no existen:

\begin{itemize}[format=\textbf]

\item

\end{itemize}

% 4 -------------------------------------------------------------------------------------------------------------
\section{}

La temperatura \textbf{T} en [Celsius] en un lugar del hemisferio norte depende de la longitud $x$, latitud $y$, y el tiempo $t$ de modo que podemos escribir $T(x, y, t)$. Mida el tiempo en horas a partir del inicio de enero.

\begin{itemize}[format=\textbf]

\item Explique qué significan las derivadas parciales

\item

\end{itemize}

% 5 -------------------------------------------------------------------------------------------------------------
\section{}

Use la definición de las \textbf{derivadas parciales} como límites para determinar $f_x(x, y), f_y(x, y)$ de la función:

\begin{itemize}[format=\textbf]

\item $$f(x, y) = xy^2 − x^3y$$

\end{itemize}

% 6 -------------------------------------------------------------------------------------------------------------
\section{}

% 7 -------------------------------------------------------------------------------------------------------------
\section{}

Determine una ecuación del \textbf{plano tangente} a la superficie dada en el punto solicitado:

\begin{itemize}[format=\textbf]

\item

\end{itemize}

% 8 -------------------------------------------------------------------------------------------------------------
\section{}

% 9 -------------------------------------------------------------------------------------------------------------
\section{}

Utilice \textbf{diferenciales} para estimar la cantidad de \textit{estaño} en una lata cerrada de estaño cuyo diámetro es 8 \textit{cm} y altura 12 \textit{cm} si el estaño tiene
0.04 \textit{cm} de espesor.

% 10 -------------------------------------------------------------------------------------------------------------
\section{}

\begin{itemize}[format=\textbf]

\item

\end{itemize}

% 11 -------------------------------------------------------------------------------------------------------------
\section{}

% 12 -------------------------------------------------------------------------------------------------------------
\section{}

% 13 -------------------------------------------------------------------------------------------------------------
\section{}

\begin{itemize}[format=\textbf]

\item

\end{itemize}

\end{document}