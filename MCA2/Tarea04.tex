\documentclass[12pt]{article}

\usepackage[spanish]{babel}

\usepackage{amsmath}
\usepackage{amssymb}

\usepackage{hyperref}
\usepackage{graphicx}
\usepackage{listings}
\usepackage{color}
\usepackage{multicol}
\usepackage{enumitem}
\usepackage{here}
\usepackage{dsfont}
\usepackage{tipa}
\usepackage{float}
\usepackage{dsfont} 
\spanishdecimal{.}

\title{Matemáticas para las Ciencias Aplicadas II}
\title{
        \textbf{Tarea 04} \\
        \vspace{1ex}
        \large Matemáticas para las Ciencias Aplicadas II \\
        Facultad de Ciencias, UNAM}
\date{\today}
\author{Flores Morán Julieta Melina \\ Zarco Romero José Antonio}

\begin{document}
\maketitle

% 1 -------------------------------------------------------------------------------------------------------------
\section{}
Verifique que la función $z = \ln{[e^x + e^y]}$ es una solución de las ecuaciones diferenciales:

\begin{itemize}[format=\textbf]

\item $\frac{\partial z}{\partial x} + \frac{\partial z}{\partial y} = 1$

  Primero calculamos las derivadas parciales de primer orden necesarias:

  Para $\frac{\partial z}{\partial x}$
  \begin{align*}
    \frac{\partial z}{\partial x}
    &= \frac{1}{e^x+e^y} \cdot \frac{\partial}{\partial x} (e^x+e^y) \\
    &= \frac{1}{e^x+e^y} \cdot e^x \\
    &= \frac{e^x}{e^x+e^y}
  \end{align*}

  Para $\frac{\partial z}{\partial y}$
  \begin{align*}
    \frac{\partial z}{\partial x}
    &= \frac{1}{e^x+e^y} \cdot \frac{\partial}{\partial y} (e^x+e^y) \\
    &= \frac{1}{e^x+e^y} \cdot e^y \\
    &= \frac{e^y}{e^x+e^y}
  \end{align*}

  Así, $\frac{\partial z}{\partial x} + \frac{\partial z}{\partial y}$
  \begin{align*}
    \frac{\partial z}{\partial x} + \frac{\partial z}{\partial y} 
    &= \frac{e^x}{e^x+e^y} + \frac{e^y}{e^x+e^y} \\
    &= \frac{e^x + e^y}{e^x+e^y} \\
    &= 1
  \end{align*}

  $\therefore z = \ln{[e^x + e^y]}$ satisface la ecuación $\frac{\partial z}{\partial x} + \frac{\partial z}{\partial y} = 1$.

\item $\frac{\partial^2 z}{\partial x^2} \frac{\partial^2 z}{\partial y^2} - \left( \frac{\partial^2 z}{\partial x \partial y} \right)^2=0$

  Primero, a partir de $\frac{\partial z}{\partial x}=\frac{e^x}{e^x+e^y}$ y $\frac{\partial z}{\partial y}=\frac{e^y}{e^x+e^y}$, calculamos las derivadas parciales de segundo orden necesarias:

  Para $\frac{\partial^2 z}{\partial^2 x}$
  \begin{align*}
    \frac{\partial^2 z}{\partial^2 x}
    &= \frac{\partial}{\partial x} \left( \frac{e^x}{e^x+e^y} \right)\\
    &= \frac{[(e^x+e^y)\cdot e^x]-[e^x \cdot e^x]}{(e^x+e^y)^2} \\
    &= \frac{e^{2x}+e^{x+y}-e^{2x}}{(e^x+e^y)^2}\\
    &= \frac{e^{x+y}}{(e^x+e^y)^2}\\
  \end{align*}

  Para $\frac{\partial^2 z}{\partial^2 y}$
  \begin{align*}
    \frac{\partial^2 z}{\partial^2 y}
    &= \frac{\partial}{\partial y} \left( \frac{e^y}{e^x+e^y} \right)\\
    &= \frac{[(e^x+e^y)\cdot e^y]-[e^y \cdot e^y]}{(e^x+e^y)^2} \\
    &= \frac{e^{x+y}+e^{2y}-e^{2y}}{(e^x+e^y)^2}\\
    &= \frac{e^{x+y}}{(e^x+e^y)^2}\\
  \end{align*}

  Para $\frac{\partial^2 z}{\partial x \partial y}$
  \begin{align*}
    \frac{\partial^2 z}{\partial x \partial y}
    &= \frac{\partial}{\partial x} \left( \frac{\partial z}{\partial y} \right) \\
    &= \frac{\partial}{\partial x} \left( \frac{e^y}{e^x+e^y} \right) \\
    &= \frac{[(e^x+e^y)\cdot 0] - (e^y \cdot e^x)}{(e^x+e^y)^2} \\
    &= - \frac{e^{x+y}}{(e^x+e^y)^2}\\
  \end{align*}

  Así, $\frac{\partial^2 z}{\partial x^2} \frac{\partial^2 z}{\partial y^2} - \left( \frac{\partial^2 z}{\partial x \partial y} \right)^2$
  \begin{align*}
    \frac{\partial^2 z}{\partial x^2} \frac{\partial^2 z}{\partial y^2} - \left( \frac{\partial^2 z}{\partial x \partial y} \right)^2
    &= \left[ \frac{e^{x+y}}{(e^x+e^y)^2}\cdot \frac{e^{x+y}}{(e^x+e^y)^2} \right] - \left[- \frac{e^{x+y}}{(e^x+e^y)^2}\right]^2 \\
    &= \frac{(e^{x+y})^2}{(e^{x+y})^4} - \frac{(e^{x+y})^2}{(e^{x+y})^4} \\
    &= 0
  \end{align*}

  $\therefore z = \ln{[e^x + e^y]}$ satisface la ecuación $\frac{\partial^2 z}{\partial x^2} \frac{\partial^2 z}{\partial y^2} - \left( \frac{\partial^2 z}{\partial x \partial y} \right)^2=0$.

\end{itemize}

% 2 -------------------------------------------------------------------------------------------------------------
\section{}

La \textit{energía cinética} de un cuerpo de masa $m$ y velocidad $v$ es $K = \frac{1}{2}mv^2$. Demuestre que $K=\frac{\partial K}{\partial m} \frac{\partial ^2 K}{\partial v^2}$

% 3 -------------------------------------------------------------------------------------------------------------
\section{}

Determine una ecuación del plano tangente a la función $z = xe^{xy}$ en el punto $(x_0, y_0) = (5, 0)$.

% 4 -------------------------------------------------------------------------------------------------------------
\section{}

Compruebe que la \textbf{aproximación lineal} en (0, 0).
$$\frac{2x+3}{4y +1} \approx 2x - 12y + 3$$

% 5 -------------------------------------------------------------------------------------------------------------
\section{}

Utilice la \textbf{regla de la cadena} para calcular $\frac{\partial z}{\partial s}$ y$\frac{\partial z}{\partial t}$. Dado que
$$z = \sin{\theta}\cos{\phi}, ~ \theta = t^2, ~ \phi = s^2t$$

% 6 -------------------------------------------------------------------------------------------------------------
\section{}

Sea $z = x^4 + x^2y$, con $x = s + 2t − u$, $y = stu^2$, utilice la \textbf{regla de la cadena} para calcular: $\frac{\partial z}{\partial s}$ $\frac{\partial z}{\partial t}$ $\frac{\partial z}{\partial u}$, donde $s=4$ $t=2$, $u=1$.

% 7 -------------------------------------------------------------------------------------------------------------
\section{}

Sea $f(x, y, z) = x^2yz - xyz^3$, $P(2, -1, 1)$, $\hat{u} = \left (0,\frac{4}{5},\frac{-3}{5} \right)$:

\begin{itemize}[format=\textbf]
  
\item Determine el \textbf{gradiente} de la función escalar $f(x, y, z)$.

\item Evalúe el \textbf{gradiente} en el punto $P$.

\item Encuentre la \textit{razón de cambio} de $f(x, y, z)$ en el punto $P$ en la dirección del vector $\hat{u}$.
\end{itemize}

% 8 -------------------------------------------------------------------------------------------------------------
\section{}

Determine la máxima \textbf{razón de cambio} de $f(x, y) = 4y\sqrt{x}$ en el punto $P(4, 1)$ y la dirección en la cuál se presenta.

% 9 -------------------------------------------------------------------------------------------------------------
\section{}

Sea $f(x, y) = x^2 + xy + y^2 + y$. Calcule los valores \textbf{máximo} y \textbf{mínimo} \textit{locales}, y \textit{punto(s)} silla de la función.

% 10 -------------------------------------------------------------------------------------------------------------
\section{}

Sea $f(x, y) = x^2 + y^2 − 2x$, donde $D$ es la región triángular cerrada con vértices $A(2, O)$, $B(O, 2)$, $C(0, -2)$.

Determine los \textbf{valores máximos absolutos}, \textbf{valores mínimos absolutos} de $f(x, y)$ sobre el conjunto $D$.

% 11 -------------------------------------------------------------------------------------------------------------
\section{}

Encuentre tres números positivos cuya suma es $100$ y cuyo producto es un máximo.

% 12 -------------------------------------------------------------------------------------------------------------
\section{}

Utilizando \textbf{multiplicadores de Lagrange}, encuentre los valores \textbf{máximo} y \textbf{mínimo} de la función sujeta a la \textbf{restricción(es)} dadas.

\begin{itemize}[format=\textbf]

\item $f(x, y) = x^2 + y^2$, sujeto a la restricción $xy = 1$.

\item $f(x, y) = xyz$, sujeto a la restricción $x^2 + 2y^2 + 3z^2 = 6$.
  
\end{itemize}

\end{document}
