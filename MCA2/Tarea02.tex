\documentclass[12pt]{article}

\usepackage[spanish]{babel}

\usepackage{amsmath}
\usepackage{amssymb}

\usepackage{hyperref}
\usepackage{graphicx}
\usepackage{listings}
\usepackage{color}
\usepackage{multicol}
\usepackage{enumitem}
\usepackage{here}
\usepackage{dsfont}
\usepackage{tipa}
\usepackage{float}
\usepackage{dsfont} 
\spanishdecimal{.}

\title{Matemáticas para las Ciencias Aplicadas II}
\title{
	\textbf{Tarea 02} \\
	\vspace{1ex}
	\large Matemáticas para las Ciencias Aplicadas II \\
	Facultad de Ciencias, UNAM}
\date{\today}
\author{Flores Morán Julieta Melina \\ Zarco Romero José Antonio}

\begin{document}
\maketitle

% 1 -------------------------------------------------------------------------------------------------------------
\section{}
Proporcione el dominio de la función vectorial.
\[
\vec{r(t)}
=
\frac{t-2}{t+2}\hat{i}
+
\sin{t}\hat{j}
+
\ln{(9-t^2)}\hat{k}
\]

El dominio de cada \textit{función componente} es como sigue:
\begin{itemize}[format=\textbf]

\item $f(t)=\frac{t-2}{t+2}$
  $$\left\{t \in \mathbb{R} ~|~ t \neq -2 \right\}$$

\item $g(t)=\sin{t}$
  $$\left\{t\in \mathbb{R} \right\}$$

\item $h(t)=\ln{(9-t^2)}$
  $$\left\{t\in \mathbb{R} ~|~ -3 < t < 3 \right\}$$

\end{itemize}
La intersección de los 3 conjuntos es $\left\{t\in \mathbb{R} ~|~ (-3<t<-2) ~ \land ~ (-2<t<3) \right\}$.

$\therefore$ El dominio de la función vectorial $\vec{r(t)}$ es $(-3,-2)$ y $(-2,3)$.


% 2 -------------------------------------------------------------------------------------------------------------
\section{}
Sea
\[
\vec{r(t)}
=
\frac{t^2-t}{t-1}\hat{i}
+
\sqrt{t+8}\hat{j}
+
\frac{\sin{\pi t}}{\ln{t}}\hat{k}
\]
Calcule $ \lim_{t \to 0} \vec{r(t)} $.

Recordemos que el límite de una función vectorial se define obteniendo los límites de sus \textit{funciones componentes}:
\begin{itemize}[format=\textbf]

\item $f(t)=\frac{t^2-t}{t-1}$
  \begin{align*}
    \lim_{t \to 0} f(t) = \lim_{t \to 0} \frac{t^2-t}{t-1} = 0
  \end{align*}

\item $g(t)=\sqrt{t+8}$
  \begin{align*}
    \lim_{t \to 0} g(t) = \lim_{t \to 0} \sqrt{t+8} = \sqrt{8} = 2\sqrt{2}
  \end{align*}

\item $h(t)=\frac{\sin{\pi t}}{\ln{t}}$
  \begin{align*}
    \lim_{t \to 0} h(t) = \lim_{t \to 0} \frac{\sin{\pi t}}{\ln{t}}
  \end{align*}

\end{itemize}

% 3 -------------------------------------------------------------------------------------------------------------
\section{}
Realice a mano la gráfica de las siguientes funciones vectoriales, indicando el sentido en que se traza la curva:
\begin{itemize}[format=\textbf]

\item $\vec{r(t)} = t^2\hat{i}+t\hat{j}+2\hat{k}$

\item $\vec{r(t)} = \cos{t}\hat{i}-\cos{t}\hat{j}+\sin{t}\hat{k}$

\end{itemize}

% 4 -------------------------------------------------------------------------------------------------------------
\section{}
Proporcione las coordenadas del punto donde se intersecta la \textbf{hélice} $\vec{r(t)}=\sin{t}\hat{i}+\cos{t}\hat{j}+t\hat{k}$, y la \textbf{esfera} $x^2+y^2+z^2=5$.

% 5 -------------------------------------------------------------------------------------------------------------
\section{}
Dibuje las proyecciones de la curva $\vec{r(t})=t\hat{i}+t\hat{j}+t^2\hat{k}$ sobre los planos $XY , XZ, YZ$. Utilice dichas proyecciones para hacer un esbozo de la curva.

% 6 -------------------------------------------------------------------------------------------------------------
\section{}
Las trayectorias de dos partículas están dadas por las siguientes funciones vectoriales:
\begin{itemize}[format=\textbf]

\item $\vec{r_1(t)}=t\hat{i}+t^2\hat{j}+t^3\hat{k}$

\item $\vec{r_2(t)}=(1+2t,1+6t,1+14t)$

\end{itemize}
¿Chocarán las partículas? ¿En qué punto? ¿Se cortarán las trayectorias?

% 7 -------------------------------------------------------------------------------------------------------------
\section{}
Proporcione las coordenadas del punto sobre la curva $\vec{r(t)}=(2\cos{t},2\sin{t},\mathrm{e}^t)$, con $t \in [0,\pi]$, donde la \textbf{recta tangente} a la curva es paralela al plano $\sqrt{3}x+y=1$.

% 8 -------------------------------------------------------------------------------------------------------------
\section{}
Proporcione las coordenadas del punto donde se intersectan las curvas:
\begin{itemize}[format=\textbf]

\item $\vec{r_1(t)}=(t,1-t,3+t^2)$

\item $\vec{r_2(s)}=(3-s,s-2,s^2)$

\end{itemize}

% 9 -------------------------------------------------------------------------------------------------------------
\section{}
Determine la \textbf{longitud de curva} para las siguientes curvas:
\begin{itemize}[format=\textbf]

\item $\vec{r(t)}=\left(2t,t^2,\frac{t^3}{3}\right)$, para $t \in [0,1]$

\item $\vec{r(t)}=\left(\cos{t},\sin{t},\ln{\cos{t}}\right)$, para $t \in \left[0,\frac{\pi}{4}\right]$

\end{itemize}

% 10 -------------------------------------------------------------------------------------------------------------
\section{}
Reparametrice la siguiente curva (respecto a la longitud de arco medida desde el punto donde $t = 0$), en la dirección en que $t$ se incrementa.
$$ \vec{r(t)}=(2t)\hat{i}+(1-3t)\hat{j}+(5+4t)\hat{k}$$

\end{document}

