\documentclass[12pt]{article}

\usepackage[spanish]{babel}
\usepackage{hyperref}
\usepackage{graphicx}
\usepackage{listings}
\usepackage{color}
\usepackage{multicol}
\usepackage{amssymb}
\spanishdecimal{.}

%% Título
\title{Matemáticas para las Ciencias Aplicadas I}

%% Fecha
\date{\today}

%% Autor
\author{Flores Morán Julieta Melina \\ Zarco Romero José Antonio}


%% Se marca el inicio del documento.
\begin{document}

%% Comando para crear el título.
\maketitle


%sección 3
\section{Problema 22}
Flores Morán Julieta Melina \\
\\
Un pequeño fabricante de electrodomésticos descubre que cuesta 9000 dólares producir 1000 tostadoras a la semana y 12000 dólares producir 1500 tostadoras a la semana.
\begin{enumerate}
\item Exprese el costo en función del número de tostadoras producidas, suponiendo que es lineal. Después, trace la gráfica.\\ 

Ya que suponemos que el costo $C$ es una función lineal del número de tostadoras $t$ de la forma $y = mx + b$, podemos escribir:

\[
	C=mt+b
\]

Utilizando los datos del problema, consideremos que conocemos los puntos (1000, 9000) y (1500, 12000). Por lo que la pendiente de la recta es

\[
	 m = \frac{\Delta {y}}{\Delta{x}} = \frac{12000-9000}{1500-1000} = \frac{3000}{500} = 6
\]

de modo que

\[
C=6t+b
\]

y su ecuación de la forma punto-pendiente $y-y_1=m(x-x_1)$, en el punto (1000, 9000) es 

\[
	C-9000=6(t-1000)
\]

o bien
\[
	C=6t+3000
\]

\[
\therefore f(t) = 6t + 3000
\]

Dicha ecuación da un posible modelo lineal, representado gráficamente como
\begin{figure}[h!]
\centering
\includegraphics[width=10cm]{img/tostadoras.png}
\end{figure}

\item  ¿Cuál es la pendiente de la gráfica y qué representa?
\\ \\
La pendiente de la recta es $m=6$ y representa la razón de cambio, en este caso el aumento, del costo de producción en dólares respecto al número de tostadoras producidas. Agregar una tostadora implica subir el costo en 6 dólares.
\\  
\item  ¿Cuál es la intersección de la gráfica con el eje $y$ y qué representa?
\\
\[
f(0) = (6 \cdot 0) + 3000 = 3000
\]
La intersección con el eje $y$ es 3000 y representa el costo inicial de producción. No depende del número de tostadoras, es una constante sumada al costo independiente del número de tostadoras.
\\
\end{enumerate}

%sección 4
\section{Problema 27}
Zarco Romero José Antonio\\
\\
La población de ciertas especies en un ambiente limitado con una población inicial de 100 y capacidad para 1 000 es 
\[
P (t) = \frac{100 000}{100 + 900 e^{-t}} 
\]
donde t se mide en años.

\begin{enumerate}
\item Grafique esta función y estime cuánto tiempo le toma a la población llegar a 900.\\
\begin{figure}[h]
\centering
\includegraphics[ width=1\textwidth]{img/graficaPob.png}
\end{figure}
\\
Aprox 4.4 años. $P(4.4) = 900.4984$

\item Encuentre la inversa de esta función y explique su significado.
\[
P = \frac{100 000}{100 + 900 e^{-t}} 
\]
Para encontrar la inversa hay que despejar t
\[
 100 + 900 e^{-t}=\frac{100 000}{P}
\]
\[
e^{-t}=\frac{100 000- 100P}{900P}
\]
\[
t=-ln (\frac{100 000- 100P}{900P} )
\]
\[
\therefore t=-ln (\frac{1 000- P}{9P} )
\]

Entonces, la función inversa es
\[
P^{-1}(t)  = -ln (\frac{1 000 - t}{9t} )
\]
y representa el tiempo requerido  para que la población de ciertas especies alcance un número t.

\item Utilice la función inversa para encontrar el tiempo
necesario para que la población llegue a 900. Compare
con el resultado del inciso \\
\[
P(900) = -ln (\frac{1 000- 900}{9(900)} )=-ln(\frac{100}{8100})=-ln(\frac{1}{81})=-(ln1-ln81)
\]
\[
=ln81-ln1=ln81-0=ln(81) = 4.3944491546724 
\]
$\approx 4.4$ años.
\end{enumerate}


\end{document}